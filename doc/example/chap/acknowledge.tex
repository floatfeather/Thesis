\chapter{致谢}

感谢北京大学和北京大学软件工程所,让我能够接触到前沿的科研项目。
浓厚的学术氛围感染了我,敦促我不断学习,打下了科研的坚实基础。

感谢熊英飞研究员,张路教授和郝丹副教授对我的指导。
我从大三开始就在张路老师的小组里学习,近五年的时间里三位老师对我耐心地指导和帮助,让我受益颇多。
从最开始的浮点数计算误差相关的研究,到自动缺陷定位的研究,
我从一个初出茅庐的计算机专业低年级学生,
变成了一个能够完成很多艰难计算机任务的高年级研究生。
老师们的培养让我在计算机基础知识,代码能力,科研能力等方面都有提升。
我也在熊英飞研究和张路教授的指导下发表了两篇CCF-A类的论文,一篇第一作者,一篇第二作者。

感谢姜佳君同学,和我一起讨论、完成这个研究。
他提出了许多宝贵的想法与建议,并且与我一起实现了\textsc{LinSD}。

感谢王博同学和臧琳飞师姐,他们的基于机器学习的缺陷修复给了本文非常多的帮助。
本文使用的从Java代码中提取特征的JDT代码来自于他们的基于机器学习的缺陷修复的代码。

最后,感谢我的父母一直陪伴着我、支持着我。他们一直是我坚强的后盾。