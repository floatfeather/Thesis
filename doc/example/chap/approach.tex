\chapter{方法}

\section{现有缺陷定位的不足}

在上一章的分析中我们发现,现有缺陷定位存在不足。

对于基于频谱的缺陷定位来说,它仅仅依赖频谱信息去区分正确语句和错误语句,
会导致很多正确语句也具有很高的怀疑度。
特别地,如果一个正确语句只被失败的测试用例覆盖,那么它将拥有非常高的怀疑度。
这是由于频谱信息的信息量太少,基于频谱的缺陷定位忽略了程序状态等被基于状态覆盖的缺陷定位关注的信息。

而基于状态覆盖的缺陷定位,虽然能够获得比频谱信息更多的信息,
但是现有的方法都依赖于大量的测试用例。
在测试用例不足的时候,基于状态覆盖的缺陷定位无法给出具有区分度的怀疑度。

\section{结合基于频谱的缺陷定位和基于状态覆盖的缺陷定位}

既然基于频谱的缺陷定位和基于状态覆盖的缺陷定位各有优劣,
那么是否可以结合这两种缺陷定位的方法呢?
事实上已经有研究\parencite{Le2016A,Xuan2014Learning},结合了多种缺陷定位方法,并且获得了比较好的结果。
\todo{展开}。
但是这些研究的结合方式都是在比较高的层次,
比如使用机器学习方法对不同缺陷定位得到的结果进行组合。
这样的结合方式会有两个缺点。
一是他们难以解释为什么他们的方法会起作用。
二是他们没有深入理解缺陷定位方法起作用的原因,仅仅是把各个方法的结果合在一起。

所以,本文试图提出一个能够结合多种缺陷定位(比如基于频谱的缺陷定位和基于状态覆盖的缺陷定位)的方法去改进缺陷定位技术,
同时本文试图解释这个结合为什么起作用的原因。

虽然在直觉上我们认为基于频谱的缺陷定位和基于状态覆盖的缺陷定位是完全不一样的。
因为基于频谱的缺陷定位依靠的是程序元素的覆盖情况,
而基于状态覆盖的缺陷定位依靠的是用谓词来划分状态。
但是事实上这两种缺陷定位技术有相似的地方。
基于频谱的缺陷定位的频谱信息,其实相当于是对每一个语句都关联了一个\mycode{true}这样的谓词。
这样看来基于频谱的缺陷定位相当于基于状态覆盖的缺陷定位的一个特殊情况。
而基于状态覆盖的缺陷定位收集的谓词的覆盖信息也可以看做是程序频谱信息的一种,
所以基于状态覆盖的缺陷定位也可以看做基于频谱的缺陷定位的一个特殊情况。

考虑\ref{sec:state_based}章中基于状态覆盖的缺陷定位的例子。
统计性调试和SOBER都无法给出很好的定位结果。
但是当观察统计性调试的覆盖情况\ref{math_2_return},
我们却可以“猜测”出当前语句很可能是错误语句。
这是因为我们带入了基于频谱的缺陷定位的假设:被失败的测试用例执行的语句,
更有可能有错误。而被通过的测试用例执行的语句,更有可能是正确的。
根据这个假设,表\ref{math_2_return}中的谓词3、4、6都不太可能是能够划分出缺陷状态的谓词,因为它们都没有被失败的测试用例覆盖过。
谓词1最有可能是能够划分出缺陷状态的谓词,其次是谓词2,最后是谓词5。
这是因为谓词1、2、5都被一个失败的测试用例覆盖过,而谓词1没有被通过的测试用例覆盖过。
这种情况下被越少的通过的测试用例覆盖,越有可能就是能够划分出缺陷状态的谓词。
怎样去具体地表示这个怀疑度呢?
这其实是基于频谱的缺陷定位解决的问题了,那就是使用怀疑度公式。
使用Ochiai怀疑度公式去计算表\ref{math_2_return}中谓词的怀疑度,得到表\ref{math_2_ochiai}。
可见谓词1以1.0000的分数远远高于其他谓词,成为怀疑度很大的谓词。
使用Ochiai怀疑度公式,计算Math的第二个缺陷的各个谓词怀疑度,
错误语句排名第3(第1到4名并列),相比于基于频谱的状态覆盖第11位、统计性调试全部为0和SOBER第10的结果,有显著提升。

\begin{table}
\centering
\begin{tabular}{|c|l|c|}
\hline
 & 谓词 & Ochiai分数\\
\hline
1 & \mycode{retValue < 0} &  1.0000 \\
\hline
2 & \mycode{retValue <= 0} &  0.7071 \\
\hline
3 & \mycode{retValue > 0} & 0.0000 \\
\hline
4 & \mycode{retValue >= 0} & 0.0000 \\
\hline
5 & \mycode{retValue != 0} & 0.4082 \\
\hline
6 & \mycode{retValue == 0} & 0.0000 \\
\hline
\end{tabular}
\caption{使用Ochiai计算谓词怀疑度,其中 \\ \mycode{retValue = (double) (getSampleSize() * getNumberOfSuccesses()) / (double) getPopulationSize()}}
\label{math_2_ochiai}
\end{table}


