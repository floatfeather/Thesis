\chapter{总结与未来工作}

本章对本文的工作进行总结,并且对未来的研究方向进行展望。

\section{总结}

本文结合了现有的缺陷定位技术,提出了结合现有缺陷定位技术的新的缺陷定位技术。
具体来说,
\begin{enumerate}
\item 本文提出了一种结合基于频谱的缺陷定位和基于状态覆盖的缺陷定位的方法~\textsc{LinPred}~,
并且在实际数据集 Defects4j 中与传统的基于频谱的缺陷定位和基于状态覆盖的缺陷定位进行了比较。
深入分析了基于频谱的缺陷定位和基于状态覆盖的缺陷定位起作用的原因,并且将其互补地结合起来。
可以发现\textsc{LinPred}在各个项目、各个公式上都有比较好的效果。
\item 本文提出了一种基于机器学习的预测谓词模型,来完善基于状态覆盖的缺陷定位中的预定义谓词。
实验效果发现,随机森林模型预测出的谓词的EXAM值比预定义谓词更优,
而神经网络模型预测出的谓词的Top-1值与预定义谓词相当。
预定义谓词与预测谓词具有互补性,两者结合后效果进一步提升。
\end{enumerate}

本文发现:
\begin{enumerate}
\item 更细粒度的数据收集会让缺陷定位效果变好。
\item 在Defects4j数据集中基于频谱的缺陷定位公式的效果优于基于状态覆盖的缺陷定位公式效果。
\item 在预定义谓词中,分支谓词对\textsc{LinPred}效果提升最明显。
本质上与使用分支覆盖的基于频谱的缺陷定位技术相同。
\item 预测谓词能够帮助更好地定位缺陷,与预定义谓词结合后效果更佳。
\end{enumerate}

\section{未来工作}

本文尝试了多种怀疑度计算公式。
统计性调试在 Defects4j 项目上效果不佳,但是并不能认为这个公式效果不好。
只是因为统计性调试不适用于失败测试用例少的场景。
所以可以考虑针对不同的应用场景,自动使用不同的怀疑度公式。
将每一个怀疑度公式的作用发挥到最好。

本文发现分支谓词起对定位缺陷起了重要作用。
同时发现了高频的预测谓词往往是和空指针的比较,和常数的比较。
可见某些类型的谓词具有更好的甄别缺陷的能力。
但是预测谓词的范围受限于项目中出现的条件表达式,
所以可能还有很多种具有明显甄别意义的谓词没有考虑到。
可以考虑增加可预测谓词的种类,来更好定位缺陷。

本文中,预测谓词取得了比较好的效果。
但是预测谓词仍有较大的提升空间,尤其是EXPR模型的准确率仍然较低。
在以后的工作中,可以考虑加入更多的特征、采样点,来提升准确率。
或者,可以考虑从另一些方面来建立模型。
比如在章\ref{sec:eval_pref_predict}中,曾经考虑过谓词是否具有积极影响。
可以尝试建立一个分类器,其分类目标是一个谓词是否具有积极影响。
