\chapter{引言}

本章主要介绍了自动缺陷定位的研究背景及其重要意义,然后阐述了本文的研究内容、主要贡献和论文结构。

\section{研究背景}

随着软件的发展,生活中越来越多的方面都与软件有着紧密的关系。
小到人们的日常出行、购物、餐饮等,大到航空航天、医药等领域,软件在人们的生活中扮演着重要的角色。
随着软件的应用领域的扩大,软件的复杂性上升,提升了软件缺陷的可能性。
软件缺陷可能会导致巨大的损失。
一个著名的被广泛引用的例子是海湾战争时,一颗导弹由于导航软件的精度缺陷而偏离了目标,导致28人死亡和100人受伤\parencite{Zou2015A}。
2002年,美国国家标准与技术研究院(NIST)发表的一篇报告\parencite{NIST2002The}显示,软件缺陷每年会导致约595亿美元的经济损失。\todo{每年这么多国内公司漏洞事件}
发现并修复软件缺陷,保障软件的高质量成为一项重要的任务。

\section{研究意义}

在发现软件缺陷之后,开发人员为了解决这个缺陷往往需要三步\parencite{Parnin2011Are}。
第一步,缺陷定位,需要找到程序中和这个缺陷有关的语句。
第二步,理解缺陷,明白为什么会发生缺陷。
第三步,修复缺陷,修改代码以让缺陷消失。
这三个步骤合起来就是调试的过程。缺陷定位作为调试的第一步,其完成速度和准确性对后面的步骤有着很大的影响。
在传统的开发环境当中,人们可以手动调试来定位缺陷,比如插入断点、打印日志信息等等。
在1989年Collofello等人就指出尝试去减少软件中的错误会花费50\%到80\%的开发和维护的精力\parencite{Collofello1989Evaluating}。
随着软件的复杂性的上升,手动地定位软件缺陷将会耗费更多开发者的时间和精力。
为了提高定位缺陷的速度,研究人员对自动化的缺陷定位展开了研究,并取得了巨大的进展\todo{cite}。
然而在2011年,Partin和Osro的一篇调查\parencite{Parnin2011Are}通过研究缺陷定位技术在实际应用场景下的效果,发现以往的评价指标并不能准确的反映缺陷定位技术在实际应用中的效果。
以往的缺陷定位技术是基于一系列关于开发人员会如何调试的假设,而这些假设在实际场景的某些情况下会失效。
自动化缺陷定位技术还有很大的发展空间。

\section{本文研究内容}

\section{本文主要贡献}

\section{论文结构}