% Copyright (c) 2008-2009 solvethis
% Copyright (c) 2010-2016 Casper Ti. Vector
% Public domain.
%
% 使用前请先仔细阅读 pkuthss 和 biblatex-caspervector 的文档,
% 特别是其中的 FAQ 部分和用红色强调的部分。
% 两者可在终端/命令提示符中用
%   texdoc pkuthss
%   texdoc biblatex-caspervector
% 调出。

% 采用了自定义的(包括大小写不同于原文件的)字体文件名,
% 并改动 ctex.cfg 等配置文件的用户请自行加入 nofonts 选项;
% 其它用户不用加入 nofonts 选项,加入之后反而会产生错误。
\documentclass[UTF8]{pkuthss}

% 使用 biblatex 排版参考文献,并规定其格式(详见 biblatex-caspervector 的文档)。
% 这里按照英文文献在前,中文文献在后排序(“sorting = ecnty”);
% 若需按照中文文献在前,英文文献在后排序,请设置“sorting = centy”;
% 若需按照引用顺序排序,请设置“sorting = none”。
% 若需在排序中实现更复杂的需求,请参考 biblatex-caspervector 的文档。
\usepackage[backend = biber, style = caspervector, utf8, sorting = ecnty]{biblatex}

% 按学校要求设定参考文献列表中的条目之内及之间的距离。
\setlength{\bibitemsep}{3bp}
% 对于 linespread 值的计算过程有兴趣的同学可以参考 pkuthss.cls。
\renewcommand*{\bibfont}{\zihao{5}\linespread{1.27}\selectfont}

% 设定文档的基本信息。
\pkuthssinfo{
	cthesisname = {硕士研究生学位论文}, ethesisname = {Doctor Thesis},
	ctitle = {测试文档}, etitle = {Test Document},
	cauthor = {某某},
	eauthor = {Test},
	studentid = {0123456789},
	date = {某年某月},
	school = {某某学院},
	cmajor = {某某专业}, emajor = {Some Major},
	direction = {某某方向},
	cmentor = {某某教授}, ementor = {Prof.\ Somebody},
	ckeywords = {其一,其二}, ekeywords = {First, Second}
}
% 载入参考文献数据库(注意不要省略“.bib”)。
\addbibresource{thesis.bib}

% 普通用户可删除此段,并相应地删除 chap/*.tex 中的
% “\pkuthssffaq % 中文测试文字。”一行。
\usepackage{color}
% \def\pkuthssffaq{%
% 	\emph{\textcolor{red}{pkuthss 文档模版最常见问题:}}

% 	\texttt{\string\cite}、\texttt{\string\parencite} %
% 	和 \texttt{\string\supercite} 三个命令分别产生%
% 	未格式化的、带方括号的和上标且带方括号的引用标记:%
% 	\cite{test-en},\parencite{test-zh}、\supercite{test-en, test-zh}。

% 	若要避免章末空白页,请在调用 pkuthss 文档类时加入 \texttt{openany} 选项。

% 	如果编译时不出参考文献,
% 	请参考 \texttt{texdoc pkuthss}“问题及其解决”一章
% 	“上游宏包可能引起的问题”一节中关于 biber 的说明。
% }

\newcommand\todo[1]{\textcolor{red}{\{TODO:#1\}}}
\newcommand\mycode[1]{{\small\ttfamily #1}}
\usepackage{hyperref}

\usepackage{listings}
\lstset{
    numbers=left, 
    numberstyle=\tiny,
    keywordstyle= \color{ blue!70},
    commentstyle= \color{red!50!green!50!blue!50}, 
    frame=shadowbox, % 阴影效果
    rulesepcolor= \color{ red!20!green!20!blue!20},
    basicstyle=\small\ttfamily
} 

\begin{document}
	% 以下为正文之前的部分,默认不进行章节编号。
	\frontmatter
	% 此后到下一 \pagestyle 命令之前不排版页眉或页脚。
	\pagestyle{empty}
	% 自动生成封面。
	\maketitle
	% 版权声明。封面要求单面打印,故需新开右页。
	\cleardoublepage
	% Copyright (c) 2008-2009 solvethis
% Copyright (c) 2010-2017 Casper Ti. Vector
% All rights reserved.
%
% Redistribution and use in source and binary forms, with or without
% modification, are permitted provided that the following conditions are
% met:
%
% * Redistributions of source code must retain the above copyright notice,
%   this list of conditions and the following disclaimer.
% * Redistributions in binary form must reproduce the above copyright
%   notice, this list of conditions and the following disclaimer in the
%   documentation and/or other materials provided with the distribution.
% * Neither the name of Peking University nor the names of its contributors
%   may be used to endorse or promote products derived from this software
%   without specific prior written permission.
%
% THIS SOFTWARE IS PROVIDED BY THE COPYRIGHT HOLDERS AND CONTRIBUTORS "AS
% IS" AND ANY EXPRESS OR IMPLIED WARRANTIES, INCLUDING, BUT NOT LIMITED TO,
% THE IMPLIED WARRANTIES OF MERCHANTABILITY AND FITNESS FOR A PARTICULAR
% PURPOSE ARE DISCLAIMED. IN NO EVENT SHALL THE COPYRIGHT HOLDER OR
% CONTRIBUTORS BE LIABLE FOR ANY DIRECT, INDIRECT, INCIDENTAL, SPECIAL,
% EXEMPLARY, OR CONSEQUENTIAL DAMAGES (INCLUDING, BUT NOT LIMITED TO,
% PROCUREMENT OF SUBSTITUTE GOODS OR SERVICES; LOSS OF USE, DATA, OR
% PROFITS; OR BUSINESS INTERRUPTION) HOWEVER CAUSED AND ON ANY THEORY OF
% LIABILITY, WHETHER IN CONTRACT, STRICT LIABILITY, OR TORT (INCLUDING
% NEGLIGENCE OR OTHERWISE) ARISING IN ANY WAY OUT OF THE USE OF THIS
% SOFTWARE, EVEN IF ADVISED OF THE POSSIBILITY OF SUCH DAMAGE.

% 此处不用 \specialchap,因为学校要求目录不包括其自己及其之前的内容。
\chapter*{版权声明}
% 综合学校的书面要求及 Word 模版来看,版权声明页不需加页眉、页脚。
\thispagestyle{empty}

任何收存和保管本论文各种版本的单位和个人,
未经本论文作者同意,不得将本论文转借他人,
亦不得随意复制、抄录、拍照或以任何方式传播。
否则一旦引起有碍作者著作权之问题,将可能承担法律责任。

% 若需排版二维码,请将二维码图片重命名为“barcode”,
% 转为合适的图片格式,并放在当前目录下,然后去掉下面 2 行的注释。
%\vfill\noindent
%\includegraphics[height = 5em]{barcode}

% vim:ts=4:sw=4


	% 此后到下一 \pagestyle 命令之前正常排版页眉和页脚。
	\cleardoublepage
	\pagestyle{plain}
	% 重置页码计数器,用大写罗马数字排版此部分页码。
	\setcounter{page}{0}
	\pagenumbering{Roman}
	% 中英文摘要。
	% Copyright (c) 2014,2016 Casper Ti. Vector
% Public domain.

\begin{cabstract}

程序调试是一个耗费时间的任务,已经有很多研究者提出了各种不同的自动缺陷定位技术去减轻手动调试的负担。
在这些自动缺陷定位技术中,基于频谱的缺陷定位和基于状态覆盖的缺陷定位是两种比较常用的缺陷定位技术。
这两种技术都是在执行时收集一些统计信息。
这两种技术存在一些相似之处,但是两种技术一直单独地发展,而它们的结合也一直没有被系统地讨论过。

本文对基于频谱的缺陷定位和基于状态覆盖的缺陷定位技术的结合进行了系统的实证研究,
并且提出一种基于机器学习的谓词预测的方法帮助缺陷定位。
本文构建了一个两种技术的统一模型,并在这个模型上系统地探索了四种变体:
不同粒度的数据收集、不同的怀疑度公式、不同的怀疑度结合方式、不同的谓词。
此外,本文还提出了一个基于机器学习的谓词预测模型,来完善基于状态覆盖的缺陷定位原有的预定义谓词。

本文的研究得到了很多结论。
第一,更细粒度的数据收集的效果远远好于粗粒度的数据收集,并且只需要花费稍微多一点的执行时间。
第二,把基于频谱的缺陷定位公式应用在基于状态覆盖的缺陷定位的预定义谓词上,其效果反而好于原有的基于状态覆盖的缺陷定位的公式。
第三,一个基于频谱的缺陷定位和基于状态覆盖的缺陷定位的线性结合模型的效果比两者都更好。
第四,结合方法的效果大部分得益于分支谓词。
第五,预测的谓词在某些指标上获得了比预定义谓词更好的结果。预测谓词与预定义谓词具有互补性,两者结合后定位效果进一步提升。

\end{cabstract}

\begin{eabstract}

Program debugging is a time-consuming task,
and researchers have proposed different kinds of automatic fault localization techniques
to mitigate the burden of manual debugging.
Among these techniques, two popular families are spectrum-based fault localization
and statistical debugging,
both localizing faults by collecting statistical information at runtime.
Though the ideas are similar, the two families have been developed independently
and their combinations have not been systematically explored.

In this paper we perform a systematical empirical study on the combination of spectrum-based
fault localization and statistical debugging,
and we propose a predicate prediction technique based on machine learning to help to locate the faults.
We first build a unified model of the two techniques,
and systematically explores four types of variations:
different granularities of data collection,
different risk evaluation formulas, different ways of combining suspiciousness scores,
and different predicates.
Then we propose a machine-learning model to predict the predicates,
instead of using pre-defined predicates in statistical debugging.

The study leads to several findings.
First, fine-grained data collection significantly outperforms
coarse-grained data collection with a little more execution overhead.
Second, the risk evaluation formulas of spectrum-based fault localization
siginificantly outperforms that of statistical debugging when used in statistical debugging.
Third, a linear combination of spectrum-based fault localization and
statistical debugging outperforms both individual approaches.
Forth, most of the effectiveness of the combined approach contributed by a simple type of predicates:
branch conditions.
Fifth, the predicted predicates are better than pre-defined predicates in some metrics.
The predicated predicates and pre-defined predicates are complementary.
The combination of them has further improvement.
\end{eabstract}

% vim:ts=4:sw=4

	% 自动生成目录。
	\tableofcontents

	% 以下为正文部分,默认要进行章节编号。
	\mainmatter
	% 序言。
	\chapter{引言}

本章主要介绍了自动缺陷定位的研究背景及其重要意义,然后阐述了本文的研究内容、主要贡献和论文结构。

\section{研究背景}

随着软件的发展,生活中越来越多的方面都与软件有着紧密的关系。
小到人们的日常出行、购物、餐饮等,大到航空航天、医药等领域,软件在人们的生活中扮演着重要的角色。
随着软件的应用领域的扩大,软件的复杂性上升,提升了软件缺陷的可能性。
软件缺陷可能会导致巨大的损失。
一个著名的被广泛引用的例子是海湾战争时,一颗导弹由于导航软件的精度缺陷而偏离了目标,导致28人死亡和100人受伤\parencite{Zou2015A}。
2002年,美国国家标准与技术研究院(NIST)发表的一篇报告\parencite{NIST2002The}显示,软件缺陷每年会导致约595亿美元的经济损失。
发现并修复软件缺陷,保障软件的高质量成为一项重要的任务。

\section{研究意义}

在发现软件缺陷之后,开发人员为了解决这个缺陷往往需要三步\parencite{Parnin2011Are}。
第一步,缺陷定位,需要找到程序中和这个缺陷有关的语句。
第二步,理解缺陷,明白为什么会发生缺陷。
第三步,修复缺陷,修改代码以让缺陷消失。
这三个步骤合起来就是调试的过程。缺陷定位作为调试的第一步,其完成速度和准确性对后面的步骤有着很大的影响。
在传统的开发环境当中,人们可以手动调试来定位缺陷,比如插入断点、打印日志信息等等。
在1989年Collofello等人就指出尝试去减少软件中的错误会花费50\%到80\%的开发和维护的精力\parencite{Collofello1989Evaluating}。
随着软件的复杂性的上升,手动地定位软件缺陷将会耗费更多开发者的时间和精力。
为了提高定位缺陷的速度,研究人员对自动化的缺陷定位展开了研究,并取得了巨大的进展
\parencite{Weiser1981Program,Weiser1984Program,Reps1997The,Ball2003From,Groce2004Understanding,
Jones2002Visualization,Liblit2005Scalable,Liu2005SOBER,
Renieres2003Fault,Abreu2006An,Xie2013A,W2009BP,
Wong2012Effective,Le2016A,Papadakis2015Metallaxis,Moon2014Ask,Zhang2006Locating,
Chandra2011Angelic,Shapiro1982Algorithmic,Zeller2002Isolating,Zeller2002Simplifying}。
然而在2011年,Partin和Osro的一篇调查\parencite{Parnin2011Are}通过研究缺陷定位技术在实际应用场景下的效果,发现以往的评价指标并不能准确的反映缺陷定位技术在实际应用中的效果。
以往的缺陷定位技术是基于一系列关于开发人员会如何调试的假设,而这些假设在实际场景的某些情况下会失效。
自动化缺陷定位技术还有很大的发展空间。

\section{本文研究内容和主要贡献}

为了能提升调试的效率,本文对自动化缺陷定位技术进行了深入的研究。
通过在实际缺陷中分析传统的缺陷定位技术的效果,本文提出了一种结合基于频谱的缺陷定位和基于状态覆盖的缺陷定位的方式。
本文探索了各种不同的结合方式的效果,分析了基于状态覆盖的缺陷定位公式的不足,并使用基于频谱的缺陷定位公式与其互补。
同时,由于基于状态覆盖的缺陷定位使用的预定义谓词的不灵活性,
本文提出了一种基于机器学习的谓词预测模型,而传统的基于状态覆盖的缺陷定位使用的是预定义谓词。
预测出的谓词与预定义谓词互补,能进一步提升定位效果。

本文的贡献如下:
\begin{itemize}
\item 在实际缺陷中深入分析了基于频谱的缺陷定位的效果,发现了基于频谱的缺陷定位利用的频谱信息粒度不够细,导致缺陷和非缺陷无法区别。
\item 在实际缺陷中深入分析了基于状态覆盖的缺陷定位的效果,发现了其怀疑度公式在实际缺陷中并不适用。
\item 提出了结合基于频谱的缺陷定位和基于状态覆盖的缺陷定位的方式。基于状态覆盖的缺陷定位的信息粒度比基于频谱的缺陷定位的信息粒度细,
而基于频谱的缺陷定位公式在实际缺陷中仍然表现良好,两者结合之后获得了更好的效果。
\item 利用结合后的模型,在实际缺陷中分析了基于状态覆盖的缺陷定位的谓词起作用的原因,发现分支是起最大作用的谓词。
\item 提出了一种基于机器学习的预测谓词的模型,能够根据语句上下文预测谓词,从而更好地定位缺陷。
\end{itemize}

\section{论文结构}

本文共七章,结构如下:

第一章为引言,介绍了本文的研究背景、研究意义、研究内容和主要贡献。

第二章为相关工作,介绍了国内外相关领域的研究现状,包括自动缺陷定位技术、机器学习技术和实验数据集三部分。

第三章为问题分析,在实际缺陷中分析了现有自动缺陷定位技术的优势和不足。

第四章介绍了基于谓词预测的结合现有技术的缺陷定位工具的设计,包括结合现有技术的方法、基于机器学习的谓词预测模型等等。

第五章介绍了基于谓词预测的结合现有技术的缺陷定位工具的实现,包括整个代码流程和各个模块的实现方式。

第六章是本文提出的缺陷定位工具的实验与验证。

第七章是对全文的总结和对未来工作的展望。


	% 各章节。
	\chapter{相关工作}

本章将介绍自动缺陷定位、机器学习的相关工作和缺陷定位所使用的数据集以帮助更好理解本文的工作。

\section{自动缺陷定位相关工作}

程序切片\parencite{Weiser1981Program,Weiser1984Program}是自动调试最早的技术之一,
但是程序切片之后可能出错的语句数量仍然比较庞大。
为了解决程序切片调试方法的短板,一种通过观察错误程序的执行特征和正确程序的执行特征的调试技术被提出。
这些技术通过收集程序执行信息,观察不同的某种特征,来定位缺陷。
比如使用路径概要\parencite{Reps1997The},反例\parencite{Ball2003From,Groce2004Understanding},语句覆盖\parencite{Jones2002Visualization}和谓词值\parencite{Liblit2005Scalable,Liu2005SOBER}等等。

本文根据北京大学熊英飞研究员对缺陷定位的分类\parencite{YingfeiFL},将缺陷定位分为以下几类。

\begin{itemize}
\item 基于切片的缺陷定位
\item 基于频谱的缺陷定位
\item 基于状态覆盖的缺陷定位
\item 基于变异的缺陷定位
\item 基于构造正确执行状态的缺陷定位
\item 基于算法式调试的缺陷定位
\item 基于差异化调试的缺陷定位
\end{itemize}

本文的研究内容主要根据基于频谱的缺陷定位和基于状态覆盖的缺陷定位。

\subsection{基于切片的缺陷定位}

Weiser在1981年提出的程序切片\parencite{Weiser1981Program,Weiser1984Program}是自动调试(特别是缺陷定位)最早的技术之一。
给定一个程序$P$和一个在$P$的语句$s$中使用的变量$v$,程序切片会找到$P$中所有可能会影响$s$中$v$的值的语句。
如果$s$中$v$的值是错误的,那么导致这个错误的缺陷语句一定在这个切片当中。
也就是说,不在这个切片当中的语句可以在调试过程中被忽略。
尽管程序切片已经减少了可能出错的语句的数量,但是切片中的语句的数量仍然比较大。
为了解决这个问题,Korel和Laski在1988年提出了动态程序切片\parencite{Korel1988Dynamic}。
动态程序切片计算某一个特定执行的切片。
后来又有很多的动态程序切片的变种被提出\parencite{Demillo1996Critical,Gyim1999An,Zhang2006Pruning,Zhang2003Precise},用于解决调试问题,并且产生了大量研究工作\parencite{Agrawal1993Debugging,Liu2007Indexing,Al2005The,Alves2011Fault,Ju2014HSFal,Wotawa2010Fault,Mao2014Slice}。

\subsection{基于频谱的缺陷定位}

基于频谱的缺陷定位是使用最广泛的自动化缺陷定位方法\parencite{YingfeiFL}。
程序频谱(~Program~ Spectrum)最早由Reps等人于1997年提出\parencite{Reps1997The},用于解决千年虫问题。
Harrold 等人在2002年\parencite{Harrold2000An}提出使用测试覆盖信息作为频谱信息的调试方法。
Renieris等人在 2003年提出使用通过的测试用例和失败的测试用例进行缺陷定位\parencite{Renieres2003Fault},奠定了此后基于频谱的缺陷定位的基础。

考虑一种极端的情况。
比如当某一个语句$s$被执行的之后,测试用例就会失败。
而通过的测试用例都不会执行语句$s$。
那么语句$s$很有可能就是导致缺陷的语句。
找出所有这样的语句$s$就可以大幅减少需要排查错误的语句。
但是,在实际的代码中这种极端的情况很少出现。
对于一个出错的语句$s$,它很可能既被失败的测试用例执行,也被通过的测试用例执行。
因为一个语句在其不同的上下文作用下会产生不同的效果。
简单地计算通过的测试用例覆盖的语句和失败的测试用例覆盖的语句的差集是无法准确找出错误语句的。
利用通过的测试用例覆盖的语句的交集和并集,与失败的测试用例覆盖的语句取差集,是最早的一种基于频谱的缺陷定位方法\parencite{Renieres2003Fault}。
这种方法也隐含着基于频谱的缺陷定位的假设:被失败的测试用例执行的语句,更有可能有错误。而被通过的测试用例执行的语句,更有可能是正确的。

Jones等人提出的Tarantula\parencite{Jones2002Visualization},直观地给开发者展示了每个语句在通过的测试用例和失败的测试用例下的参与情况。
参与情况也被称为怀疑度。
% 每条语句的参与情况,使用公式
% $$
% \mathrm{Tarantula}(s) = \frac{\frac{a_{ep}}{a_{p}}}{\frac{a_{ep}}{a_{p}} + \frac{a_{ef}}{a_{f}}}
% $$
% 计算。
% 这个公式计算的值也被称为怀疑度。
怀疑度更高的语句会在怀疑列表更靠前的位置。
相比于交集并集差集的方法,Tarantula在Siemens数据集上可以将错误的语句放在怀疑列表更前面的位置\parencite{Jones2005Empirical}。

Tarantula之后,又有很多计算怀疑度的公式被提出。
效果比较好的Ochiai由Abreu等人提出\parencite{Abreu2006An}。
% $$
% \mathrm{Ochiai}(s) = \frac{a_{ef}}{\sqrt{a_{f} \times (a_{ef} + a_{ep})}}
% $$
Ochiai由\parencite{Meyer2004Comparison}提出用于计算基因的相似度。
Abreu等人将其引入用于计算怀疑度,并与Jaccard\parencite{Chen2002Pinpoint},Tarantula,AMPLE\parencite{Dallmeier2005Lightweight}比较,发现Ochiai计算的怀疑度使得定位效果更好\parencite{Abreu2006An,Abreu2007On}。
此后Xie等人在理论上证明了不存在单一最佳公式\parencite{Xie2013A},
对怀疑度公式的研究一直没有停下。

除了直接提出用于计算的公式之外,研究人员也开始使用机器学习的方法去学习怀疑度的公式。
Wong等人提出使用反向传播神经网络来定位缺陷\parencite{W2009BP}。
使用的输入数据是频谱信息(语句覆盖信息)和对应的测试用例是通过还是失败。
输入数据每一行对应一个测试用例。
第i列为1表示的是该测试用例覆盖了第i个语句,为0则表示没有覆盖。
预测的标签为1表示该测试用例失败了,为0表示通过了。
为了减少需要分析的可能出错的语句的个数(每一行输入数据的维度),优先使用所有失败的测试用例覆盖的语句。
此后Wong又提出了使用径向基核函数的神经网络来定位缺陷\parencite{Wong2012Effective}。

\subsection{基于状态覆盖的缺陷定位}

在缺陷定位的时候,定位的程序元素的大小也会影响结果。程序元素可以是一条语句,一个方法,一个文件。
程序元素的粒度越细,对测试信息的利用越精确。
然而单个元素上覆盖的测试数量越少,统计显著性越低。
如果把程序的每个执行状态作为程序元素,那么这会是一个比语句更加精细的粒度。
定位结果也将更加精细,对测试的利用也会更加充分。
但是,几乎不会有两个测试覆盖完全相同的状态,因为一个状态所包含的上下文信息往往十分复杂,很难完全一致。
于是使用抽象状态代替具体状态。
使用谓词将具体状态划分为抽象状态。
谓词是形如\mycode{a $>$ 0}这样的条件表达式。

Liblit等人最早提出了预定义谓词来划分状态\parencite{Liblit2005Scalable},并提出了统计性调试。
通过预定义在哪些代码结构中插入哪些谓词,统计性调试能够收集到许多抽象状态的覆盖情况。
% 利用表\ref{state_symbol}中的数学符号,统计性调试的公式可以表达为
% $$
% \mathrm{StatisticalDebugging}(s) = \frac{2}{\frac{1}{\frac{t_f}{t_f + t_p} - \frac{a_f}{a_f + a_p}} + \frac{log(F)}{log(t_f)}}
% $$

Liu等人改进了计算公式,提出了SOBER\parencite{Liu2006Statistical}。
虽然Liblit的方法可以有效定位一些错误,但是Liblit的方法只考虑了一个谓词是否在一次执行中为真,
而没有考虑为真的次数。
SOBER提出新的计算公式,从概率分布的角度来计算怀疑度。
% 公式计算的是对一个谓词,在失败的测试用例下这个谓词为真的概率分布,和在通过的测试用例下这个谓词为真的概率分布是否相似。
% 如果概率分布无论是在失败的测试用例中还是通过的测试用例中都一样,那么这个谓词对应的变量等和缺陷的关系就越小。
% 如果两个概率分布相差很大,说明这个谓词对应的抽象状态很有可能就有缺陷状态。
% 引入这个缺陷状态的语句很可能就是出错的语句。

除了预定义谓词以外,研究人员还提出各种从程序中获取谓词的方法。
Le等人提出 Savant\parencite{Le2016A} ,使用程序中的不变式的变化来划分状态。
程序中的不变式使用 Daikon\parencite{Ernst2007The} 挖掘。
Savant使用Learning-to-rank方法,通过分析经典的怀疑度分数和在通过的测试用例和失败的测试用例上观察到的不变式,来定位错误的方法。
Savant基于三个出发点。一,在失败的测试用例和通过的测试用例中表现出不同的不变式的程序元素,被怀疑是有错误的。
二,如果这些程序元素拥有很高的经典的怀疑度分数,那么它们更有可能是错误的。
三,有一些不变式比其他不变式更加可疑,比如\mycode{ x == null}。
% 而 Savant 的工作并没有引用 Liblit \parencite{Liblit2005Scalable}和 Liu \parencite{Liu2006Statistical},
% 很可能是在不知道统计性调试的情况下完成的。

\subsection{基于变异的缺陷定位}

变异是对程序的任意随机修改,由变异算子得到。
变异分析是测试领域的一个概念,被用于衡量一个测试集的好坏。
变异分析在程序中插入变异,得到很多变异体,然后使用一组测试去执行变异体。
如果一个测试集中任意测试在一个变异体上得到不同的结果,那么这个变异体被这个测试杀死。
能杀死越多变异体的测试集越好。

变异被引入缺陷定位,用于定位缺陷。
Papadakis等人提出Metallaxis\parencite{Papadakis2015Metallaxis},一个基于变异的缺陷定位。
Metallaxis基于两个假设:
\begin{itemize}
\item 当变异和错误在一个程序的同一条语句上时,失败的测试用例输出发生变化的概率大于通过的测试用例输出发生变化的概率。
\item 当变异和错误不在同一条语句上时,通过测试用例输出发生变化的概率大于失败的测试用例输出发生变化的概率。
\end{itemize}
基于表\ref{mutant_symbol},Metallaxis的怀疑度计算公式为
$$
\mathrm{Metallaxis}(m) = \frac{m_f}{\sqrt{F \times (m_f + m_p)}}
$$
与Ochiai类似。

Moon等人提出另一个基于变异的缺陷定位技术MUSE\parencite{Moon2014Ask}。
MUSE利用变异分析去捕捉单个语句和观察到的缺陷之间的关系。
MUSE基于的两个假设是:
\begin{itemize}
\item 一个失败的测试用例,比起在变异了正确语句的变异体上,在变异了错误语句的变异体上更容易变成通过的。
\item 一个通过的测试用例,比起在变异了失败语句的变异体上,在变异了正确语句的变异体上更容易变成失败的。
\end{itemize}
基于表\ref{mutant_symbol},MUSE的怀疑度计算公式为
$$
\mathrm{MUSE}(m) = m_{f2p} - m_{p2f} \times \frac{\sum_{m}^{}{m_{f2p}}}{\sum_{m}^{}{m_{p2f}}}
$$

\begin{table}
\centering
\caption{基于变异的缺陷定位的数学符号及其意义}
\begin{tabular}{|c|c|}
\hline
$m$ & 变异体 \\
\hline
$m_f$ & 变异m导致输出发生变化的失败的测试用例个数 \\
\hline
$m_p$ & 变异m导致输出发生变化的通过的测试用例个数 \\
\hline
$m_{f2p}$ & 变异m导致失败的测试用例变成通过的测试用例的个数 \\
\hline
$m_{p2f}$ & 变异m导致通过的测试用例变成失败的测试用例的个数 \\
\hline
$F$ & 失败的测试用例的个数 \\
\hline
\end{tabular}
\label{mutant_symbol}
\end{table}

\subsection{基于构造正确执行状态的缺陷定位}

MUSE通过变异体,可以把失败的测试用例变成通过,通过的测试用例变成失败的。
假如有一个变异体,它可以把失败的测试用例变成通过的,且不会影响通过的测试用例,那么这个变异体很可能就是缺陷的补丁。
但是直接分析出这样的变异体是很困难的。

Zhang提出的谓词翻转\parencite{Zhang2006Locating}巧妙地避免了直接分析出正确的补丁,而是使用改变程序状态来达到相同的目的。
假如出错的是一个布尔表达式,改变程序中一个布尔表达式的取值(把真变成假,或者把假变成真),强制改变执行的分支。
假如谓词翻转后,失败的测试用例变成通过的,那么对应的布尔表达式很可能有错误。

谓词翻转是局限在布尔表达式,天使调试\parencite{Chandra2011Angelic}则试图解决任意表达式的错误。
天使调试要求同时具有天使性和灵活性。
天使性是指,存在常量c(天使值)把表达式的求值结果替换成c,失败的测试变得通过。
灵活性是指,对于一个修复的候选$e$,和一个通过测试用例输入$I_p$,如果把$e$
的值替换成一个不同的值(不同于$I_p$下$e$的值),这个测试仍然通过。
利用符号执行约束求解计算得到天使值。
也由于符号执行的开销,天使调试无法应用到大型程序上。

\subsection{基于算法式调试的缺陷定位}

Shapiro提出的算法式调试\parencite{Shapiro1982Algorithmic},通过对子问题询问“是”或“否”来定位缺陷。
算法式调试把复杂的计算步骤拆为小的子问题。
算法式调试的一个问题是,子问题的正确结果可能是不知道的。
如果是让人进行交互式地判断,那么人需要花费时间计算判断子问题的结果。

\subsection{基于差异化调试的缺陷定位}

差异化调试由Zeller等人提出\parencite{Zeller2002Isolating,Zeller2002Simplifying}。
不同于以往的使用动态分析或静态分析的方法去关注源代码,
差异化调试关注程序状态,特别地,差异化调试关注当程序没有出错时的程序状态和程序出错时的程序状态。
差异化调试尝试找到一个最小的修改集合,当把这个集合应用到没有出错时的程序状态后,程序出错了。

\section{机器学习相关工作}

近年来,机器学习相关的工作在高速地发展中。
机器学习也越来越多地被应用到软件工程的领域,并且发挥着重要的作用。
神经网络是一种常用的机器学习模型。
神经网络的定义多种多样,采用 Kohonen 1998年在期刊《Neural Networks》上的定义,为
“神经网络是由具有适应性的简单单元组成的广泛并行互连的网络,它的组织能够模拟生物神经系统对真是世界物体所作出的交互反应”。
本文使用的误差后向传播网络能够近似复杂的非线性函数\parencite{Hecht1992Theory}。
Neumann等人提出一种结合了主成分分析和误差后向传播网络的软件风险分析\parencite{Neumann2002An}。
Tadayon使用神经网络做软件代价评估。
在缺陷定位方面,也有很多神经网络的应用。
比如之前提到的Wong的两篇工作\parencite{W2009BP,Wong2012Effective}。

\section{缺陷定位的数据集}

要研究缺陷定位,需要一个包含缺陷的数据集。
这个数据集一般来说需要有多个缺陷。
对每一个缺陷,会有对应的测试用例,和对应的一个正确的版本。
这些测试用例中既有通过的,也必定有失败的。
失败的这个测试用例就是由缺陷导致。

Siemens数据集\parencite{Hutchins1994Experiments}是一个很早的数据集,用于测试充分性的实验。
它由七个C程序组成,大小在141行到512行之间。
这七个C程序衍生出132个有缺陷的C程序。
每一个错误版本会恰好有一个缺陷。
这个缺陷可能涉及多行甚至多个文件。
但是这些程序的缺陷是由作者手动插入的,根据作者的描述其实和一个简单的变异操作非常相似。
Siemens数据集的输入被构造用于实现完全的代码覆盖。
尽管它一开始并不是被用于缺陷定位,但是很多缺陷定位技术都使用它来验证效果,比如最早的基于频谱的缺陷定位方法\parencite{Renieres2003Fault},Ochiai\parencite{Abreu2006An,Abreu2007On},
SOBER\parencite{Liu2006Statistical},BPNN\parencite{W2009BP}等等。

Defects4j数据集\parencite{Just2014Defects4J}是一个真实、独立、可重现缺陷的数据集。
它的v1.0版本由五个Java开源项目的357个缺陷组成(该数据集仍在更新当中)。
每一个错误版本会恰好有一个缺陷。
这个缺陷可能涉及多行甚至多个文件。
与Siemens数据集相比,~Defects4j~数据集的缺陷和测试用例都更接近实际开发情况。
Savant\parencite{Le2016A}就是在Defects4j数据集上验证的。

	\chapter{研究背景}

\section{缺陷定位例子}

\section{使用基于频谱的缺陷定位}

\section{使用基于状态覆盖的缺陷定位}

比如,对于如下代码C代码,
\lstset{language=C}
\begin{lstlisting}
a = abs(a);
if (update_b) {
    b = sqrt(a);
}
\end{lstlisting}
当\mycode{a}和\mycode{b}的类型都为\mycode{int}时,如果\mycode{a}的值为最小的\mycode{int}时(\mycode{a = -2147483648}),
则代码会在第3行出错(\mycode{b}的值为\mycode{NAN})。
这是因为当\mycode{a = -2147483648}时,第1行的\mycode{a}会被赋值为一个负数,于是在第3行进行\mycode{sqrt}操作的时候,就被出错。
在第1行的时候,考虑两个抽象的状态\mycode{a $\ge$ 0}和\mycode{a $<$ 0}。
发现通过的测试只有\mycode{a $\ge$ 0}这个状态,而失败的测试只有\mycode{a $<$ 0}这个状态。
所以可以认为\mycode{a $<$ 0}是缺陷状态,引入这个状态的第1行的语句很可能就是缺陷语句。
	% 结论。
	% Copyright (c) 2014,2016 Casper Ti. Vector
% Public domain.

\specialchap{结论}
% \pkuthssffaq % 中文测试文字。

% vim:ts=4:sw=4


	% 正文中的附录部分。
	\appendix
	% 排版参考文献列表。bibintoc 选项使“参考文献”出现在目录中;
	% 如果同时要使参考文献列表参与章节编号,可将“bibintoc”改为“bibnumbered”。
	\printbibliography[heading = bibintoc]
	% \bibliographystyle{IEEEtran}%

	% \bibliography{thesis.bib}
	% 各附录。
	% Copyright (c) 2014,2016 Casper Ti. Vector
% Public domain.

\chapter{附件}
% \pkuthssffaq % 中文测试文字。

% vim:ts=4:sw=4


	% 以下为正文之后的部分,默认不进行章节编号。
	\backmatter
	% 致谢。
	\chapter{致谢}

感谢北京大学和北京大学软件工程所,让我能够接触到前沿的科研项目。
浓厚的学术氛围感染了我,敦促我不断学习,打下了科研的坚实基础。

感谢熊英飞研究员,张路教授和郝丹副教授对我的指导。
我从大三开始就在张路老师的小组里学习,近五年的时间里三位老师对我耐心地指导和帮助,让我受益颇多。
从最开始的浮点数计算误差相关的研究,到自动缺陷定位的研究,
我从一个初出茅庐的计算机专业低年级学生,
变成了一个能够完成很多艰难计算机任务的高年级研究生。
老师们的培养让我在计算机基础知识,代码能力,科研能力等方面都有提升。
我也在熊英飞研究和张路教授的指导下发表了两篇CCF-A类的论文,一篇第一作者,一篇第二作者。

感谢姜佳君同学,和我一起讨论、完成这个研究。
他提出了许多宝贵的想法与建议,并且与我一起实现了\textsc{LinSD}。

感谢王博同学和臧琳飞师姐,他们的基于机器学习的缺陷修复给了本文非常多的帮助。
本文使用的从Java代码中提取特征的JDT代码来自于他们的基于机器学习的缺陷修复的代码。

最后,感谢我的父母一直陪伴着我、支持着我。他们一直是我坚强的后盾。
	% 原创性声明和使用授权说明。
	% Copyright (c) 2008-2009 solvethis
% Copyright (c) 2010-2017 Casper Ti. Vector
% All rights reserved.
%
% Redistribution and use in source and binary forms, with or without
% modification, are permitted provided that the following conditions are
% met:
%
% * Redistributions of source code must retain the above copyright notice,
%   this list of conditions and the following disclaimer.
% * Redistributions in binary form must reproduce the above copyright
%   notice, this list of conditions and the following disclaimer in the
%   documentation and/or other materials provided with the distribution.
% * Neither the name of Peking University nor the names of its contributors
%   may be used to endorse or promote products derived from this software
%   without specific prior written permission.
%
% THIS SOFTWARE IS PROVIDED BY THE COPYRIGHT HOLDERS AND CONTRIBUTORS "AS
% IS" AND ANY EXPRESS OR IMPLIED WARRANTIES, INCLUDING, BUT NOT LIMITED TO,
% THE IMPLIED WARRANTIES OF MERCHANTABILITY AND FITNESS FOR A PARTICULAR
% PURPOSE ARE DISCLAIMED. IN NO EVENT SHALL THE COPYRIGHT HOLDER OR
% CONTRIBUTORS BE LIABLE FOR ANY DIRECT, INDIRECT, INCIDENTAL, SPECIAL,
% EXEMPLARY, OR CONSEQUENTIAL DAMAGES (INCLUDING, BUT NOT LIMITED TO,
% PROCUREMENT OF SUBSTITUTE GOODS OR SERVICES; LOSS OF USE, DATA, OR
% PROFITS; OR BUSINESS INTERRUPTION) HOWEVER CAUSED AND ON ANY THEORY OF
% LIABILITY, WHETHER IN CONTRACT, STRICT LIABILITY, OR TORT (INCLUDING
% NEGLIGENCE OR OTHERWISE) ARISING IN ANY WAY OUT OF THE USE OF THIS
% SOFTWARE, EVEN IF ADVISED OF THE POSSIBILITY OF SUCH DAMAGE.

{
	\ctexset{section = {
		format+ = {\centering}, beforeskip = {40bp}, afterskip = {15bp}
	}}

	% 学校书面要求本页面不要页码,但在给出的 Word 模版中又有页码且编入了目录。
	% 此处以 Word 模版为实际标准进行设定。
	\specialchap{北京大学学位论文原创性声明和使用授权说明}
	\mbox{}\vspace*{-3em}
	\section*{原创性声明}

	本人郑重声明:
	所呈交的学位论文,是本人在导师的指导下,独立进行研究工作所取得的成果。
	除文中已经注明引用的内容外,
	本论文不含任何其他个人或集体已经发表或撰写过的作品或成果。
	对本文的研究做出重要贡献的个人和集体,均已在文中以明确方式标明。
	本声明的法律结果由本人承担。
	\vskip 1em
	\rightline{%
		论文作者签名:\hspace{5em}%
		日期:\hspace{2em}年\hspace{2em}月\hspace{2em}日%
	}

	\section*{%
		学位论文使用授权说明\\[-0.33em]
		\textmd{\zihao{5}(必须装订在提交学校图书馆的印刷本)}%
	}

	本人完全了解北京大学关于收集、保存、使用学位论文的规定,即:
	\begin{itemize}
		\item 按照学校要求提交学位论文的印刷本和电子版本;
		\item 学校有权保存学位论文的印刷本和电子版,
			并提供目录检索与阅览服务,在校园网上提供服务;
		\item 学校可以采用影印、缩印、数字化或其它复制手段保存论文;
		\item 因某种特殊原因需要延迟发布学位论文电子版,
			授权学校在 $\Box$\nobreakspace{}一年 /
			$\Box$\nobreakspace{}两年 /
			$\Box$\nobreakspace{}三年以后在校园网上全文发布。
	\end{itemize}
	\centerline{(保密论文在解密后遵守此规定)}
	\vskip 1em
	\rightline{%
		论文作者签名:\hspace{5em}导师签名:\hspace{5em}%
		日期:\hspace{2em}年\hspace{2em}月\hspace{2em}日%
	}

	% 若需排版二维码,请将二维码图片重命名为“barcode”,
	% 转为合适的图片格式,并放在当前目录下,然后去掉下面 2 行的注释。
	%\vfill\noindent
	%\includegraphics[height = 5em]{barcode}
}

% vim:ts=4:sw=4

\end{document}

% vim:ts=4:sw=4
