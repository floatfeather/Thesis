% Copyright (c) 2008-2009 solvethis
% Copyright (c) 2010-2016 Casper Ti. Vector
% Public domain.
%
% 使用前请先仔细阅读 pkuthss 和 biblatex-caspervector 的文档,
% 特别是其中的 FAQ 部分和用红色强调的部分。
% 两者可在终端/命令提示符中用
%   texdoc pkuthss
%   texdoc biblatex-caspervector
% 调出。

% 采用了自定义的(包括大小写不同于原文件的)字体文件名,
% 并改动 ctex.cfg 等配置文件的用户请自行加入 nofonts 选项;
% 其它用户不用加入 nofonts 选项,加入之后反而会产生错误。
\documentclass[UTF8]{pkuthss}

% 使用 biblatex 排版参考文献,并规定其格式(详见 biblatex-caspervector 的文档)。
% 这里按照英文文献在前,中文文献在后排序(“sorting = ecnty”);
% 若需按照中文文献在前,英文文献在后排序,请设置“sorting = centy”;
% 若需按照引用顺序排序,请设置“sorting = none”。
% 若需在排序中实现更复杂的需求,请参考 biblatex-caspervector 的文档。
\usepackage[backend = biber, style = caspervector, utf8, sorting = ecnty]{biblatex}

% 按学校要求设定参考文献列表中的条目之内及之间的距离。
\setlength{\bibitemsep}{3bp}
% 对于 linespread 值的计算过程有兴趣的同学可以参考 pkuthss.cls。
\renewcommand*{\bibfont}{\zihao{5}\linespread{1.27}\selectfont}

% 设定文档的基本信息。
\pkuthssinfo{
	cthesisname = {博士研究生学位论文}, ethesisname = {Doctor Thesis},
	ctitle = {测试文档}, etitle = {Test Document},
	cauthor = {某某},
	eauthor = {Test},
	studentid = {0123456789},
	date = {某年某月},
	school = {某某学院},
	cmajor = {某某专业}, emajor = {Some Major},
	direction = {某某方向},
	cmentor = {某某教授}, ementor = {Prof.\ Somebody},
	ckeywords = {其一,其二}, ekeywords = {First, Second}
}
% 载入参考文献数据库(注意不要省略“.bib”)。
\addbibresource{thesis.bib}

% 普通用户可删除此段,并相应地删除 chap/*.tex 中的
% “\pkuthssffaq % 中文测试文字。”一行。
\usepackage{color}
\def\pkuthssffaq{%
	\emph{\textcolor{red}{pkuthss 文档模版最常见问题:}}

	\texttt{\string\cite}、\texttt{\string\parencite} %
	和 \texttt{\string\supercite} 三个命令分别产生%
	未格式化的、带方括号的和上标且带方括号的引用标记:%
	\cite{test-en},\parencite{test-zh}、\supercite{test-en, test-zh}。

	若要避免章末空白页,请在调用 pkuthss 文档类时加入 \texttt{openany} 选项。

	如果编译时不出参考文献,
	请参考 \texttt{texdoc pkuthss}“问题及其解决”一章
	“上游宏包可能引起的问题”一节中关于 biber 的说明。
}

\begin{document}
	% 以下为正文之前的部分,默认不进行章节编号。
	\frontmatter
	% 此后到下一 \pagestyle 命令之前不排版页眉或页脚。
	\pagestyle{empty}
	% 自动生成封面。
	\maketitle
	% 版权声明。封面要求单面打印,故需新开右页。
	\cleardoublepage
	% Copyright (c) 2008-2009 solvethis
% Copyright (c) 2010-2017 Casper Ti. Vector
% All rights reserved.
%
% Redistribution and use in source and binary forms, with or without
% modification, are permitted provided that the following conditions are
% met:
%
% * Redistributions of source code must retain the above copyright notice,
%   this list of conditions and the following disclaimer.
% * Redistributions in binary form must reproduce the above copyright
%   notice, this list of conditions and the following disclaimer in the
%   documentation and/or other materials provided with the distribution.
% * Neither the name of Peking University nor the names of its contributors
%   may be used to endorse or promote products derived from this software
%   without specific prior written permission.
%
% THIS SOFTWARE IS PROVIDED BY THE COPYRIGHT HOLDERS AND CONTRIBUTORS "AS
% IS" AND ANY EXPRESS OR IMPLIED WARRANTIES, INCLUDING, BUT NOT LIMITED TO,
% THE IMPLIED WARRANTIES OF MERCHANTABILITY AND FITNESS FOR A PARTICULAR
% PURPOSE ARE DISCLAIMED. IN NO EVENT SHALL THE COPYRIGHT HOLDER OR
% CONTRIBUTORS BE LIABLE FOR ANY DIRECT, INDIRECT, INCIDENTAL, SPECIAL,
% EXEMPLARY, OR CONSEQUENTIAL DAMAGES (INCLUDING, BUT NOT LIMITED TO,
% PROCUREMENT OF SUBSTITUTE GOODS OR SERVICES; LOSS OF USE, DATA, OR
% PROFITS; OR BUSINESS INTERRUPTION) HOWEVER CAUSED AND ON ANY THEORY OF
% LIABILITY, WHETHER IN CONTRACT, STRICT LIABILITY, OR TORT (INCLUDING
% NEGLIGENCE OR OTHERWISE) ARISING IN ANY WAY OUT OF THE USE OF THIS
% SOFTWARE, EVEN IF ADVISED OF THE POSSIBILITY OF SUCH DAMAGE.

% 此处不用 \specialchap,因为学校要求目录不包括其自己及其之前的内容。
\chapter*{版权声明}
% 综合学校的书面要求及 Word 模版来看,版权声明页不需加页眉、页脚。
\thispagestyle{empty}

任何收存和保管本论文各种版本的单位和个人,
未经本论文作者同意,不得将本论文转借他人,
亦不得随意复制、抄录、拍照或以任何方式传播。
否则一旦引起有碍作者著作权之问题,将可能承担法律责任。

% 若需排版二维码,请将二维码图片重命名为“barcode”,
% 转为合适的图片格式,并放在当前目录下,然后去掉下面 2 行的注释。
%\vfill\noindent
%\includegraphics[height = 5em]{barcode}

% vim:ts=4:sw=4


	% 此后到下一 \pagestyle 命令之前正常排版页眉和页脚。
	\cleardoublepage
	\pagestyle{plain}
	% 重置页码计数器,用大写罗马数字排版此部分页码。
	\setcounter{page}{0}
	\pagenumbering{Roman}
	% 中英文摘要。
	% Copyright (c) 2014,2016 Casper Ti. Vector
% Public domain.

\begin{cabstract}

程序调试是一个耗费时间的任务,已经有很多研究者提出了各种不同的自动缺陷定位技术去减轻手动调试的负担。
在这些自动缺陷定位技术中,基于频谱的缺陷定位和基于状态覆盖的缺陷定位是两种比较常用的缺陷定位技术。
这两种技术都是在执行时收集一些统计信息。
这两种技术存在一些相似之处,但是两种技术一直单独地发展,而它们的结合也一直没有被系统地讨论过。

本文对基于频谱的缺陷定位和基于状态覆盖的缺陷定位技术的结合进行了系统的实证研究,
并且提出一种基于机器学习的谓词预测的方法帮助缺陷定位。
本文构建了一个两种技术的统一模型,并在这个模型上系统地探索了四种变体:
不同粒度的数据收集、不同的怀疑度公式、不同的怀疑度结合方式、不同的谓词。
此外,本文还提出了一个基于机器学习的谓词预测模型,来完善基于状态覆盖的缺陷定位原有的预定义谓词。

本文的研究得到了很多结论。
第一,更细粒度的数据收集的效果远远好于粗粒度的数据收集,并且只需要花费稍微多一点的执行时间。
第二,把基于频谱的缺陷定位公式应用在基于状态覆盖的缺陷定位的预定义谓词上,其效果反而好于原有的基于状态覆盖的缺陷定位的公式。
第三,一个基于频谱的缺陷定位和基于状态覆盖的缺陷定位的线性结合模型的效果比两者都更好。
第四,结合方法的效果大部分得益于分支谓词。
第五,预测的谓词在某些指标上获得了比预定义谓词更好的结果。预测谓词与预定义谓词具有互补性,两者结合后定位效果进一步提升。

\end{cabstract}

\begin{eabstract}

Program debugging is a time-consuming task,
and researchers have proposed different kinds of automatic fault localization techniques
to mitigate the burden of manual debugging.
Among these techniques, two popular families are spectrum-based fault localization
and statistical debugging,
both localizing faults by collecting statistical information at runtime.
Though the ideas are similar, the two families have been developed independently
and their combinations have not been systematically explored.

In this paper we perform a systematical empirical study on the combination of spectrum-based
fault localization and statistical debugging,
and we propose a predicate prediction technique based on machine learning to help to locate the faults.
We first build a unified model of the two techniques,
and systematically explores four types of variations:
different granularities of data collection,
different risk evaluation formulas, different ways of combining suspiciousness scores,
and different predicates.
Then we propose a machine-learning model to predict the predicates,
instead of using pre-defined predicates in statistical debugging.

The study leads to several findings.
First, fine-grained data collection significantly outperforms
coarse-grained data collection with a little more execution overhead.
Second, the risk evaluation formulas of spectrum-based fault localization
siginificantly outperforms that of statistical debugging when used in statistical debugging.
Third, a linear combination of spectrum-based fault localization and
statistical debugging outperforms both individual approaches.
Forth, most of the effectiveness of the combined approach contributed by a simple type of predicates:
branch conditions.
Fifth, the predicted predicates are better than pre-defined predicates in some metrics.
The predicated predicates and pre-defined predicates are complementary.
The combination of them has further improvement.
\end{eabstract}

% vim:ts=4:sw=4

	% 自动生成目录。
	\tableofcontents

	% 以下为正文部分,默认要进行章节编号。
	\mainmatter
	% 序言。
	\chapter{引言}

本章主要介绍了自动缺陷定位的研究背景及其重要意义,然后阐述了本文的研究内容、主要贡献和论文结构。

\section{研究背景}

随着软件的发展,生活中越来越多的方面都与软件有着紧密的关系。
小到人们的日常出行、购物、餐饮等,大到航空航天、医药等领域,软件在人们的生活中扮演着重要的角色。
随着软件的应用领域的扩大,软件的复杂性上升,提升了软件缺陷的可能性。
软件缺陷可能会导致巨大的损失。
一个著名的被广泛引用的例子是海湾战争时,一颗导弹由于导航软件的精度缺陷而偏离了目标,导致28人死亡和100人受伤\parencite{Zou2015A}。
2002年,美国国家标准与技术研究院(NIST)发表的一篇报告\parencite{NIST2002The}显示,软件缺陷每年会导致约595亿美元的经济损失。
发现并修复软件缺陷,保障软件的高质量成为一项重要的任务。

\section{研究意义}

在发现软件缺陷之后,开发人员为了解决这个缺陷往往需要三步\parencite{Parnin2011Are}。
第一步,缺陷定位,需要找到程序中和这个缺陷有关的语句。
第二步,理解缺陷,明白为什么会发生缺陷。
第三步,修复缺陷,修改代码以让缺陷消失。
这三个步骤合起来就是调试的过程。缺陷定位作为调试的第一步,其完成速度和准确性对后面的步骤有着很大的影响。
在传统的开发环境当中,人们可以手动调试来定位缺陷,比如插入断点、打印日志信息等等。
在1989年Collofello等人就指出尝试去减少软件中的错误会花费50\%到80\%的开发和维护的精力\parencite{Collofello1989Evaluating}。
随着软件的复杂性的上升,手动地定位软件缺陷将会耗费更多开发者的时间和精力。
为了提高定位缺陷的速度,研究人员对自动化的缺陷定位展开了研究,并取得了巨大的进展
\parencite{Weiser1981Program,Weiser1984Program,Reps1997The,Ball2003From,Groce2004Understanding,
Jones2002Visualization,Liblit2005Scalable,Liu2005SOBER,
Renieres2003Fault,Abreu2006An,Xie2013A,W2009BP,
Wong2012Effective,Le2016A,Papadakis2015Metallaxis,Moon2014Ask,Zhang2006Locating,
Chandra2011Angelic,Shapiro1982Algorithmic,Zeller2002Isolating,Zeller2002Simplifying}。
然而在2011年,Partin和Osro的一篇调查\parencite{Parnin2011Are}通过研究缺陷定位技术在实际应用场景下的效果,发现以往的评价指标并不能准确的反映缺陷定位技术在实际应用中的效果。
以往的缺陷定位技术是基于一系列关于开发人员会如何调试的假设,而这些假设在实际场景的某些情况下会失效。
自动化缺陷定位技术还有很大的发展空间。

\section{本文研究内容和主要贡献}

为了能提升调试的效率,本文对自动化缺陷定位技术进行了深入的研究。
通过在实际缺陷中分析传统的缺陷定位技术的效果,本文提出了一种结合基于频谱的缺陷定位和基于状态覆盖的缺陷定位的方式。
本文探索了各种不同的结合方式的效果,分析了基于状态覆盖的缺陷定位公式的不足,并使用基于频谱的缺陷定位公式与其互补。
同时,由于基于状态覆盖的缺陷定位使用的预定义谓词的不灵活性,
本文提出了一种基于机器学习的谓词预测模型,而传统的基于状态覆盖的缺陷定位使用的是预定义谓词。
预测出的谓词与预定义谓词互补,能进一步提升定位效果。

本文的贡献如下:
\begin{itemize}
\item 在实际缺陷中深入分析了基于频谱的缺陷定位的效果,发现了基于频谱的缺陷定位利用的频谱信息粒度不够细,导致缺陷和非缺陷无法区别。
\item 在实际缺陷中深入分析了基于状态覆盖的缺陷定位的效果,发现了其怀疑度公式在实际缺陷中并不适用。
\item 提出了结合基于频谱的缺陷定位和基于状态覆盖的缺陷定位的方式。基于状态覆盖的缺陷定位的信息粒度比基于频谱的缺陷定位的信息粒度细,
而基于频谱的缺陷定位公式在实际缺陷中仍然表现良好,两者结合之后获得了更好的效果。
\item 利用结合后的模型,在实际缺陷中分析了基于状态覆盖的缺陷定位的谓词起作用的原因,发现分支是起最大作用的谓词。
\item 提出了一种基于机器学习的预测谓词的模型,能够根据语句上下文预测谓词,从而更好地定位缺陷。
\end{itemize}

\section{论文结构}

本文共七章,结构如下:

第一章为引言,介绍了本文的研究背景、研究意义、研究内容和主要贡献。

第二章为相关工作,介绍了国内外相关领域的研究现状,包括自动缺陷定位技术、机器学习技术和实验数据集三部分。

第三章为问题分析,在实际缺陷中分析了现有自动缺陷定位技术的优势和不足。

第四章介绍了基于谓词预测的结合现有技术的缺陷定位工具的设计,包括结合现有技术的方法、基于机器学习的谓词预测模型等等。

第五章介绍了基于谓词预测的结合现有技术的缺陷定位工具的实现,包括整个代码流程和各个模块的实现方式。

第六章是本文提出的缺陷定位工具的实验与验证。

第七章是对全文的总结和对未来工作的展望。


	% 各章节。
	% Copyright (c) 2014,2016 Casper Ti. Vector
% Public domain.

\chapter{章节}
% \pkuthssffaq % 中文测试文字。

% vim:ts=4:sw=4

	% 结论。
	% Copyright (c) 2014,2016 Casper Ti. Vector
% Public domain.

\specialchap{结论}
% \pkuthssffaq % 中文测试文字。

% vim:ts=4:sw=4


	% 正文中的附录部分。
	\appendix
	% 排版参考文献列表。bibintoc 选项使“参考文献”出现在目录中;
	% 如果同时要使参考文献列表参与章节编号,可将“bibintoc”改为“bibnumbered”。
	\printbibliography[heading = bibintoc]
	% 各附录。
	% Copyright (c) 2014,2016 Casper Ti. Vector
% Public domain.

\chapter{附件}
% \pkuthssffaq % 中文测试文字。

% vim:ts=4:sw=4


	% 以下为正文之后的部分,默认不进行章节编号。
	\backmatter
	% 致谢。
	\chapter{致谢}

感谢北京大学和北京大学软件工程所,让我能够接触到前沿的科研项目。
浓厚的学术氛围感染了我,敦促我不断学习,打下了科研的坚实基础。

感谢熊英飞研究员,张路教授和郝丹副教授对我的指导。
我从大三开始就在张路老师的小组里学习,近五年的时间里三位老师对我耐心地指导和帮助,让我受益颇多。
从最开始的浮点数计算误差相关的研究,到自动缺陷定位的研究,
我从一个初出茅庐的计算机专业低年级学生,
变成了一个能够完成很多艰难计算机任务的高年级研究生。
老师们的培养让我在计算机基础知识,代码能力,科研能力等方面都有提升。
我也在熊英飞研究和张路教授的指导下发表了两篇CCF-A类的论文,一篇第一作者,一篇第二作者。

感谢姜佳君同学,和我一起讨论、完成这个研究。
他提出了许多宝贵的想法与建议,并且与我一起实现了\textsc{LinSD}。

感谢王博同学和臧琳飞师姐,他们的基于机器学习的缺陷修复给了本文非常多的帮助。
本文使用的从Java代码中提取特征的JDT代码来自于他们的基于机器学习的缺陷修复的代码。

最后,感谢我的父母一直陪伴着我、支持着我。他们一直是我坚强的后盾。
	% 原创性声明和使用授权说明。
	% Copyright (c) 2008-2009 solvethis
% Copyright (c) 2010-2017 Casper Ti. Vector
% All rights reserved.
%
% Redistribution and use in source and binary forms, with or without
% modification, are permitted provided that the following conditions are
% met:
%
% * Redistributions of source code must retain the above copyright notice,
%   this list of conditions and the following disclaimer.
% * Redistributions in binary form must reproduce the above copyright
%   notice, this list of conditions and the following disclaimer in the
%   documentation and/or other materials provided with the distribution.
% * Neither the name of Peking University nor the names of its contributors
%   may be used to endorse or promote products derived from this software
%   without specific prior written permission.
%
% THIS SOFTWARE IS PROVIDED BY THE COPYRIGHT HOLDERS AND CONTRIBUTORS "AS
% IS" AND ANY EXPRESS OR IMPLIED WARRANTIES, INCLUDING, BUT NOT LIMITED TO,
% THE IMPLIED WARRANTIES OF MERCHANTABILITY AND FITNESS FOR A PARTICULAR
% PURPOSE ARE DISCLAIMED. IN NO EVENT SHALL THE COPYRIGHT HOLDER OR
% CONTRIBUTORS BE LIABLE FOR ANY DIRECT, INDIRECT, INCIDENTAL, SPECIAL,
% EXEMPLARY, OR CONSEQUENTIAL DAMAGES (INCLUDING, BUT NOT LIMITED TO,
% PROCUREMENT OF SUBSTITUTE GOODS OR SERVICES; LOSS OF USE, DATA, OR
% PROFITS; OR BUSINESS INTERRUPTION) HOWEVER CAUSED AND ON ANY THEORY OF
% LIABILITY, WHETHER IN CONTRACT, STRICT LIABILITY, OR TORT (INCLUDING
% NEGLIGENCE OR OTHERWISE) ARISING IN ANY WAY OUT OF THE USE OF THIS
% SOFTWARE, EVEN IF ADVISED OF THE POSSIBILITY OF SUCH DAMAGE.

{
	\ctexset{section = {
		format+ = {\centering}, beforeskip = {40bp}, afterskip = {15bp}
	}}

	% 学校书面要求本页面不要页码,但在给出的 Word 模版中又有页码且编入了目录。
	% 此处以 Word 模版为实际标准进行设定。
	\specialchap{北京大学学位论文原创性声明和使用授权说明}
	\mbox{}\vspace*{-3em}
	\section*{原创性声明}

	本人郑重声明:
	所呈交的学位论文,是本人在导师的指导下,独立进行研究工作所取得的成果。
	除文中已经注明引用的内容外,
	本论文不含任何其他个人或集体已经发表或撰写过的作品或成果。
	对本文的研究做出重要贡献的个人和集体,均已在文中以明确方式标明。
	本声明的法律结果由本人承担。
	\vskip 1em
	\rightline{%
		论文作者签名:\hspace{5em}%
		日期:\hspace{2em}年\hspace{2em}月\hspace{2em}日%
	}

	\section*{%
		学位论文使用授权说明\\[-0.33em]
		\textmd{\zihao{5}(必须装订在提交学校图书馆的印刷本)}%
	}

	本人完全了解北京大学关于收集、保存、使用学位论文的规定,即:
	\begin{itemize}
		\item 按照学校要求提交学位论文的印刷本和电子版本;
		\item 学校有权保存学位论文的印刷本和电子版,
			并提供目录检索与阅览服务,在校园网上提供服务;
		\item 学校可以采用影印、缩印、数字化或其它复制手段保存论文;
		\item 因某种特殊原因需要延迟发布学位论文电子版,
			授权学校在 $\Box$\nobreakspace{}一年 /
			$\Box$\nobreakspace{}两年 /
			$\Box$\nobreakspace{}三年以后在校园网上全文发布。
	\end{itemize}
	\centerline{(保密论文在解密后遵守此规定)}
	\vskip 1em
	\rightline{%
		论文作者签名:\hspace{5em}导师签名:\hspace{5em}%
		日期:\hspace{2em}年\hspace{2em}月\hspace{2em}日%
	}

	% 若需排版二维码,请将二维码图片重命名为“barcode”,
	% 转为合适的图片格式,并放在当前目录下,然后去掉下面 2 行的注释。
	%\vfill\noindent
	%\includegraphics[height = 5em]{barcode}
}

% vim:ts=4:sw=4

\end{document}

% vim:ts=4:sw=4
