% Copyright (c) 2008-2009 solvethis
% Copyright (c) 2010-2016 Casper Ti. Vector
% Public domain.
%
% 使用前请先仔细阅读 pkuthss 和 biblatex-caspervector 的文档,
% 特别是其中的 FAQ 部分和用红色强调的部分。
% 两者可在终端/命令提示符中用
%   texdoc pkuthss
%   texdoc biblatex-caspervector
% 调出。

% 采用了自定义的(包括大小写不同于原文件的)字体文件名,
% 并改动 ctex.cfg 等配置文件的用户请自行加入 nofonts 选项;
% 其它用户不用加入 nofonts 选项,加入之后反而会产生错误。
\documentclass[UTF8]{pkuthss}

% 使用 biblatex 排版参考文献,并规定其格式(详见 biblatex-caspervector 的文档)。
% 这里按照英文文献在前,中文文献在后排序(“sorting = ecnty”);
% 若需按照中文文献在前,英文文献在后排序,请设置“sorting = centy”;
% 若需按照引用顺序排序,请设置“sorting = none”。
% 若需在排序中实现更复杂的需求,请参考 biblatex-caspervector 的文档。
\usepackage[backend = biber, style = caspervector, utf8, sorting = ecnty]{biblatex}

% 按学校要求设定参考文献列表中的条目之内及之间的距离。
\setlength{\bibitemsep}{3bp}
% 对于 linespread 值的计算过程有兴趣的同学可以参考 pkuthss.cls。
\renewcommand*{\bibfont}{\zihao{5}\linespread{1.27}\selectfont}

% 设定文档的基本信息。
\pkuthssinfo{
	cthesisname = {硕士研究生学位论文}, ethesisname = {Doctor Thesis},
	ctitle = {测试文档}, etitle = {Test Document},
	cauthor = {某某},
	eauthor = {Test},
	studentid = {0123456789},
	date = {某年某月},
	school = {某某学院},
	cmajor = {某某专业}, emajor = {Some Major},
	direction = {某某方向},
	cmentor = {某某教授}, ementor = {Prof.\ Somebody},
	ckeywords = {其一,其二}, ekeywords = {First, Second}
}
% 载入参考文献数据库(注意不要省略“.bib”)。
\addbibresource{thesis.bib}

% 普通用户可删除此段,并相应地删除 chap/*.tex 中的
% “\pkuthssffaq % 中文测试文字。”一行。
\usepackage{color}
% \def\pkuthssffaq{%
% 	\emph{\textcolor{red}{pkuthss 文档模版最常见问题:}}

% 	\texttt{\string\cite}、\texttt{\string\parencite} %
% 	和 \texttt{\string\supercite} 三个命令分别产生%
% 	未格式化的、带方括号的和上标且带方括号的引用标记:%
% 	\cite{test-en},\parencite{test-zh}、\supercite{test-en, test-zh}。

% 	若要避免章末空白页,请在调用 pkuthss 文档类时加入 \texttt{openany} 选项。

% 	如果编译时不出参考文献,
% 	请参考 \texttt{texdoc pkuthss}“问题及其解决”一章
% 	“上游宏包可能引起的问题”一节中关于 biber 的说明。
% }

\newcommand\todo[1]{\textcolor{red}{\{TODO:#1\}}}
\newcommand\mycode[1]{{\small\ttfamily #1}}
\usepackage{hyperref}

\usepackage{listings}
\lstset{
    numbers=left, 
    numberstyle=\tiny,
    keywordstyle= \color{ blue!70},
    commentstyle= \color{red!50!green!50!blue!50}, 
    frame=shadowbox, % 阴影效果
    rulesepcolor= \color{ red!20!green!20!blue!20},
    basicstyle=\small\ttfamily,
    breaklines
} 

\begin{document}
	% 以下为正文之前的部分,默认不进行章节编号。
	\frontmatter
	% 此后到下一 \pagestyle 命令之前不排版页眉或页脚。
	\pagestyle{empty}
	% 自动生成封面。
	\maketitle
	% 版权声明。封面要求单面打印,故需新开右页。
	\cleardoublepage
	% Copyright (c) 2008-2009 solvethis
% Copyright (c) 2010-2017 Casper Ti. Vector
% All rights reserved.
%
% Redistribution and use in source and binary forms, with or without
% modification, are permitted provided that the following conditions are
% met:
%
% * Redistributions of source code must retain the above copyright notice,
%   this list of conditions and the following disclaimer.
% * Redistributions in binary form must reproduce the above copyright
%   notice, this list of conditions and the following disclaimer in the
%   documentation and/or other materials provided with the distribution.
% * Neither the name of Peking University nor the names of its contributors
%   may be used to endorse or promote products derived from this software
%   without specific prior written permission.
%
% THIS SOFTWARE IS PROVIDED BY THE COPYRIGHT HOLDERS AND CONTRIBUTORS "AS
% IS" AND ANY EXPRESS OR IMPLIED WARRANTIES, INCLUDING, BUT NOT LIMITED TO,
% THE IMPLIED WARRANTIES OF MERCHANTABILITY AND FITNESS FOR A PARTICULAR
% PURPOSE ARE DISCLAIMED. IN NO EVENT SHALL THE COPYRIGHT HOLDER OR
% CONTRIBUTORS BE LIABLE FOR ANY DIRECT, INDIRECT, INCIDENTAL, SPECIAL,
% EXEMPLARY, OR CONSEQUENTIAL DAMAGES (INCLUDING, BUT NOT LIMITED TO,
% PROCUREMENT OF SUBSTITUTE GOODS OR SERVICES; LOSS OF USE, DATA, OR
% PROFITS; OR BUSINESS INTERRUPTION) HOWEVER CAUSED AND ON ANY THEORY OF
% LIABILITY, WHETHER IN CONTRACT, STRICT LIABILITY, OR TORT (INCLUDING
% NEGLIGENCE OR OTHERWISE) ARISING IN ANY WAY OUT OF THE USE OF THIS
% SOFTWARE, EVEN IF ADVISED OF THE POSSIBILITY OF SUCH DAMAGE.

% 此处不用 \specialchap,因为学校要求目录不包括其自己及其之前的内容。
\chapter*{版权声明}
% 综合学校的书面要求及 Word 模版来看,版权声明页不需加页眉、页脚。
\thispagestyle{empty}

任何收存和保管本论文各种版本的单位和个人,
未经本论文作者同意,不得将本论文转借他人,
亦不得随意复制、抄录、拍照或以任何方式传播。
否则一旦引起有碍作者著作权之问题,将可能承担法律责任。

% 若需排版二维码,请将二维码图片重命名为“barcode”,
% 转为合适的图片格式,并放在当前目录下,然后去掉下面 2 行的注释。
%\vfill\noindent
%\includegraphics[height = 5em]{barcode}

% vim:ts=4:sw=4


	% 此后到下一 \pagestyle 命令之前正常排版页眉和页脚。
	\cleardoublepage
	\pagestyle{plain}
	% 重置页码计数器,用大写罗马数字排版此部分页码。
	\setcounter{page}{0}
	\pagenumbering{Roman}
	% 中英文摘要。
	% Copyright (c) 2014,2016 Casper Ti. Vector
% Public domain.

\begin{cabstract}

程序调试是一个耗费时间的任务,已经有很多研究者提出了各种不同的自动缺陷定位技术去减轻手动调试的负担。
在这些自动缺陷定位技术中,基于频谱的缺陷定位和基于状态覆盖的缺陷定位是两种比较常用的缺陷定位技术。
这两种技术都是在执行时收集一些统计信息。
这两种技术存在一些相似之处,但是两种技术一直单独地发展,而它们的结合也一直没有被系统地讨论过。

本文对基于频谱的缺陷定位和基于状态覆盖的缺陷定位技术的结合进行了系统的实证研究,
并且提出一种基于机器学习的谓词预测的方法帮助缺陷定位。
本文构建了一个两种技术的统一模型,并在这个模型上系统地探索了四种变体:
不同粒度的数据收集、不同的怀疑度公式、不同的怀疑度结合方式、不同的谓词。
此外,本文还提出了一个基于机器学习的谓词预测模型,来完善基于状态覆盖的缺陷定位原有的预定义谓词。

本文的研究得到了很多结论。
第一,更细粒度的数据收集的效果远远好于粗粒度的数据收集,并且只需要花费稍微多一点的执行时间。
第二,把基于频谱的缺陷定位公式应用在基于状态覆盖的缺陷定位的预定义谓词上,其效果反而好于原有的基于状态覆盖的缺陷定位的公式。
第三,一个基于频谱的缺陷定位和基于状态覆盖的缺陷定位的线性结合模型的效果比两者都更好。
第四,结合方法的效果大部分得益于分支谓词。
第五,预测的谓词在某些指标上获得了比预定义谓词更好的结果。预测谓词与预定义谓词具有互补性,两者结合后定位效果进一步提升。

\end{cabstract}

\begin{eabstract}

Program debugging is a time-consuming task,
and researchers have proposed different kinds of automatic fault localization techniques
to mitigate the burden of manual debugging.
Among these techniques, two popular families are spectrum-based fault localization
and statistical debugging,
both localizing faults by collecting statistical information at runtime.
Though the ideas are similar, the two families have been developed independently
and their combinations have not been systematically explored.

In this paper we perform a systematical empirical study on the combination of spectrum-based
fault localization and statistical debugging,
and we propose a predicate prediction technique based on machine learning to help to locate the faults.
We first build a unified model of the two techniques,
and systematically explores four types of variations:
different granularities of data collection,
different risk evaluation formulas, different ways of combining suspiciousness scores,
and different predicates.
Then we propose a machine-learning model to predict the predicates,
instead of using pre-defined predicates in statistical debugging.

The study leads to several findings.
First, fine-grained data collection significantly outperforms
coarse-grained data collection with a little more execution overhead.
Second, the risk evaluation formulas of spectrum-based fault localization
siginificantly outperforms that of statistical debugging when used in statistical debugging.
Third, a linear combination of spectrum-based fault localization and
statistical debugging outperforms both individual approaches.
Forth, most of the effectiveness of the combined approach contributed by a simple type of predicates:
branch conditions.
Fifth, the predicted predicates are better than pre-defined predicates in some metrics.
The predicated predicates and pre-defined predicates are complementary.
The combination of them has further improvement.
\end{eabstract}

% vim:ts=4:sw=4

	% 自动生成目录。
	\tableofcontents

	% 以下为正文部分,默认要进行章节编号。
	\mainmatter
	% 序言。
	\chapter{引言}

本章主要介绍了自动缺陷定位的研究背景及其重要意义,然后阐述了本文的研究内容、主要贡献和论文结构。

\section{研究背景}

随着软件的发展,生活中越来越多的方面都与软件有着紧密的关系。
小到人们的日常出行、购物、餐饮等,大到航空航天、医药等领域,软件在人们的生活中扮演着重要的角色。
随着软件的应用领域的扩大,软件的复杂性上升,提升了软件缺陷的可能性。
软件缺陷可能会导致巨大的损失。
一个著名的被广泛引用的例子是海湾战争时,一颗导弹由于导航软件的精度缺陷而偏离了目标,导致28人死亡和100人受伤\parencite{Zou2015A}。
2002年,美国国家标准与技术研究院(NIST)发表的一篇报告\parencite{NIST2002The}显示,软件缺陷每年会导致约595亿美元的经济损失。
发现并修复软件缺陷,保障软件的高质量成为一项重要的任务。

\section{研究意义}

在发现软件缺陷之后,开发人员为了解决这个缺陷往往需要三步\parencite{Parnin2011Are}。
第一步,缺陷定位,需要找到程序中和这个缺陷有关的语句。
第二步,理解缺陷,明白为什么会发生缺陷。
第三步,修复缺陷,修改代码以让缺陷消失。
这三个步骤合起来就是调试的过程。缺陷定位作为调试的第一步,其完成速度和准确性对后面的步骤有着很大的影响。
在传统的开发环境当中,人们可以手动调试来定位缺陷,比如插入断点、打印日志信息等等。
在1989年Collofello等人就指出尝试去减少软件中的错误会花费50\%到80\%的开发和维护的精力\parencite{Collofello1989Evaluating}。
随着软件的复杂性的上升,手动地定位软件缺陷将会耗费更多开发者的时间和精力。
为了提高定位缺陷的速度,研究人员对自动化的缺陷定位展开了研究,并取得了巨大的进展
\parencite{Weiser1981Program,Weiser1984Program,Reps1997The,Ball2003From,Groce2004Understanding,
Jones2002Visualization,Liblit2005Scalable,Liu2005SOBER,
Renieres2003Fault,Abreu2006An,Xie2013A,W2009BP,
Wong2012Effective,Le2016A,Papadakis2015Metallaxis,Moon2014Ask,Zhang2006Locating,
Chandra2011Angelic,Shapiro1982Algorithmic,Zeller2002Isolating,Zeller2002Simplifying}。
然而在2011年,Partin和Osro的一篇调查\parencite{Parnin2011Are}通过研究缺陷定位技术在实际应用场景下的效果,发现以往的评价指标并不能准确的反映缺陷定位技术在实际应用中的效果。
以往的缺陷定位技术是基于一系列关于开发人员会如何调试的假设,而这些假设在实际场景的某些情况下会失效。
自动化缺陷定位技术还有很大的发展空间。

\section{本文研究内容和主要贡献}

为了能提升调试的效率,本文对自动化缺陷定位技术进行了深入的研究。
通过在实际缺陷中分析传统的缺陷定位技术的效果,本文提出了一种结合基于频谱的缺陷定位和基于状态覆盖的缺陷定位的方式。
本文探索了各种不同的结合方式的效果,分析了基于状态覆盖的缺陷定位公式的不足,并使用基于频谱的缺陷定位公式与其互补。
同时,由于基于状态覆盖的缺陷定位使用的预定义谓词的不灵活性,
本文提出了一种基于机器学习的谓词预测模型,而传统的基于状态覆盖的缺陷定位使用的是预定义谓词。
预测出的谓词与预定义谓词互补,能进一步提升定位效果。

本文的贡献如下:
\begin{itemize}
\item 在实际缺陷中深入分析了基于频谱的缺陷定位的效果,发现了基于频谱的缺陷定位利用的频谱信息粒度不够细,导致缺陷和非缺陷无法区别。
\item 在实际缺陷中深入分析了基于状态覆盖的缺陷定位的效果,发现了其怀疑度公式在实际缺陷中并不适用。
\item 提出了结合基于频谱的缺陷定位和基于状态覆盖的缺陷定位的方式。基于状态覆盖的缺陷定位的信息粒度比基于频谱的缺陷定位的信息粒度细,
而基于频谱的缺陷定位公式在实际缺陷中仍然表现良好,两者结合之后获得了更好的效果。
\item 利用结合后的模型,在实际缺陷中分析了基于状态覆盖的缺陷定位的谓词起作用的原因,发现分支是起最大作用的谓词。
\item 提出了一种基于机器学习的预测谓词的模型,能够根据语句上下文预测谓词,从而更好地定位缺陷。
\end{itemize}

\section{论文结构}

本文共七章,结构如下:

第一章为引言,介绍了本文的研究背景、研究意义、研究内容和主要贡献。

第二章为相关工作,介绍了国内外相关领域的研究现状,包括自动缺陷定位技术、机器学习技术和实验数据集三部分。

第三章为问题分析,在实际缺陷中分析了现有自动缺陷定位技术的优势和不足。

第四章介绍了基于谓词预测的结合现有技术的缺陷定位工具的设计,包括结合现有技术的方法、基于机器学习的谓词预测模型等等。

第五章介绍了基于谓词预测的结合现有技术的缺陷定位工具的实现,包括整个代码流程和各个模块的实现方式。

第六章是本文提出的缺陷定位工具的实验与验证。

第七章是对全文的总结和对未来工作的展望。


	% 各章节。
	\chapter{相关工作}

本章将介绍自动缺陷定位、机器学习的相关工作和缺陷定位所使用的数据集以帮助更好理解本文的工作。

\section{自动缺陷定位相关工作}

程序切片\parencite{Weiser1981Program,Weiser1984Program}是自动调试最早的技术之一,
但是程序切片之后可能出错的语句数量仍然比较庞大。
为了解决程序切片调试方法的短板,一种通过观察错误程序的执行特征和正确程序的执行特征的调试技术被提出。
这些技术通过收集程序执行信息,观察不同的某种特征,来定位缺陷。
比如使用路径概要\parencite{Reps1997The},反例\parencite{Ball2003From,Groce2004Understanding},语句覆盖\parencite{Jones2002Visualization}和谓词值\parencite{Liblit2005Scalable,Liu2005SOBER}等等。

本文根据北京大学熊英飞研究员对缺陷定位的分类\parencite{YingfeiFL},将缺陷定位分为以下几类。

\begin{itemize}
\item 基于切片的缺陷定位
\item 基于频谱的缺陷定位
\item 基于状态覆盖的缺陷定位
\item 基于变异的缺陷定位
\item 基于构造正确执行状态的缺陷定位
\item 基于算法式调试的缺陷定位
\item 基于差异化调试的缺陷定位
\end{itemize}

本文的研究内容主要根据基于频谱的缺陷定位和基于状态覆盖的缺陷定位。

\subsection{基于切片的缺陷定位}

Weiser在1981年提出的程序切片\parencite{Weiser1981Program,Weiser1984Program}是自动调试(特别是缺陷定位)最早的技术之一。
给定一个程序$P$和一个在$P$的语句$s$中使用的变量$v$,程序切片会找到$P$中所有可能会影响$s$中$v$的值的语句。
如果$s$中$v$的值是错误的,那么导致这个错误的缺陷语句一定在这个切片当中。
也就是说,不在这个切片当中的语句可以在调试过程中被忽略。
尽管程序切片已经减少了可能出错的语句的数量,但是切片中的语句的数量仍然比较大。
为了解决这个问题,Korel和Laski在1988年提出了动态程序切片\parencite{Korel1988Dynamic}。
动态程序切片计算某一个特定执行的切片。
后来又有很多的动态程序切片的变种被提出\parencite{Demillo1996Critical,Gyim1999An,Zhang2006Pruning,Zhang2003Precise},用于解决调试问题,并且产生了大量研究工作\parencite{Agrawal1993Debugging,Liu2007Indexing,Al2005The,Alves2011Fault,Ju2014HSFal,Wotawa2010Fault,Mao2014Slice}。

\subsection{基于频谱的缺陷定位}

基于频谱的缺陷定位是使用最广泛的自动化缺陷定位方法\parencite{YingfeiFL}。
程序频谱(~Program~ Spectrum)最早由Reps等人于1997年提出\parencite{Reps1997The},用于解决千年虫问题。
Harrold 等人在2002年\parencite{Harrold2000An}提出使用测试覆盖信息作为频谱信息的调试方法。
Renieris等人在 2003年提出使用通过的测试用例和失败的测试用例进行缺陷定位\parencite{Renieres2003Fault},奠定了此后基于频谱的缺陷定位的基础。

考虑一种极端的情况。
比如当某一个语句$s$被执行的之后,测试用例就会失败。
而通过的测试用例都不会执行语句$s$。
那么语句$s$很有可能就是导致缺陷的语句。
找出所有这样的语句$s$就可以大幅减少需要排查错误的语句。
但是,在实际的代码中这种极端的情况很少出现。
对于一个出错的语句$s$,它很可能既被失败的测试用例执行,也被通过的测试用例执行。
因为一个语句在其不同的上下文作用下会产生不同的效果。
简单地计算通过的测试用例覆盖的语句和失败的测试用例覆盖的语句的差集是无法准确找出错误语句的。
利用通过的测试用例覆盖的语句的交集和并集,与失败的测试用例覆盖的语句取差集,是最早的一种基于频谱的缺陷定位方法\parencite{Renieres2003Fault}。
这种方法也隐含着基于频谱的缺陷定位的假设:被失败的测试用例执行的语句,更有可能有错误。而被通过的测试用例执行的语句,更有可能是正确的。

Jones等人提出的Tarantula\parencite{Jones2002Visualization},直观地给开发者展示了每个语句在通过的测试用例和失败的测试用例下的参与情况。
参与情况也被称为怀疑度。
% 每条语句的参与情况,使用公式
% $$
% \mathrm{Tarantula}(s) = \frac{\frac{a_{ep}}{a_{p}}}{\frac{a_{ep}}{a_{p}} + \frac{a_{ef}}{a_{f}}}
% $$
% 计算。
% 这个公式计算的值也被称为怀疑度。
怀疑度更高的语句会在怀疑列表更靠前的位置。
相比于交集并集差集的方法,Tarantula在Siemens数据集上可以将错误的语句放在怀疑列表更前面的位置\parencite{Jones2005Empirical}。

Tarantula之后,又有很多计算怀疑度的公式被提出。
效果比较好的Ochiai由Abreu等人提出\parencite{Abreu2006An}。
% $$
% \mathrm{Ochiai}(s) = \frac{a_{ef}}{\sqrt{a_{f} \times (a_{ef} + a_{ep})}}
% $$
Ochiai由\parencite{Meyer2004Comparison}提出用于计算基因的相似度。
Abreu等人将其引入用于计算怀疑度,并与Jaccard\parencite{Chen2002Pinpoint},Tarantula,AMPLE\parencite{Dallmeier2005Lightweight}比较,发现Ochiai计算的怀疑度使得定位效果更好\parencite{Abreu2006An,Abreu2007On}。
此后Xie等人在理论上证明了不存在单一最佳公式\parencite{Xie2013A},
对怀疑度公式的研究一直没有停下。

除了直接提出用于计算的公式之外,研究人员也开始使用机器学习的方法去学习怀疑度的公式。
Wong等人提出使用反向传播神经网络来定位缺陷\parencite{W2009BP}。
使用的输入数据是频谱信息(语句覆盖信息)和对应的测试用例是通过还是失败。
输入数据每一行对应一个测试用例。
第i列为1表示的是该测试用例覆盖了第i个语句,为0则表示没有覆盖。
预测的标签为1表示该测试用例失败了,为0表示通过了。
为了减少需要分析的可能出错的语句的个数(每一行输入数据的维度),优先使用所有失败的测试用例覆盖的语句。
此后Wong又提出了使用径向基核函数的神经网络来定位缺陷\parencite{Wong2012Effective}。

\subsection{基于状态覆盖的缺陷定位}

在缺陷定位的时候,定位的程序元素的大小也会影响结果。程序元素可以是一条语句,一个方法,一个文件。
程序元素的粒度越细,对测试信息的利用越精确。
然而单个元素上覆盖的测试数量越少,统计显著性越低。
如果把程序的每个执行状态作为程序元素,那么这会是一个比语句更加精细的粒度。
定位结果也将更加精细,对测试的利用也会更加充分。
但是,几乎不会有两个测试覆盖完全相同的状态,因为一个状态所包含的上下文信息往往十分复杂,很难完全一致。
于是使用抽象状态代替具体状态。
使用谓词将具体状态划分为抽象状态。
谓词是形如\mycode{a $>$ 0}这样的条件表达式。

Liblit等人最早提出了预定义谓词来划分状态\parencite{Liblit2005Scalable},并提出了统计性调试。
通过预定义在哪些代码结构中插入哪些谓词,统计性调试能够收集到许多抽象状态的覆盖情况。
% 利用表\ref{state_symbol}中的数学符号,统计性调试的公式可以表达为
% $$
% \mathrm{StatisticalDebugging}(s) = \frac{2}{\frac{1}{\frac{t_f}{t_f + t_p} - \frac{a_f}{a_f + a_p}} + \frac{log(F)}{log(t_f)}}
% $$

Liu等人改进了计算公式,提出了SOBER\parencite{Liu2006Statistical}。
虽然Liblit的方法可以有效定位一些错误,但是Liblit的方法只考虑了一个谓词是否在一次执行中为真,
而没有考虑为真的次数。
SOBER提出新的计算公式,从概率分布的角度来计算怀疑度。
% 公式计算的是对一个谓词,在失败的测试用例下这个谓词为真的概率分布,和在通过的测试用例下这个谓词为真的概率分布是否相似。
% 如果概率分布无论是在失败的测试用例中还是通过的测试用例中都一样,那么这个谓词对应的变量等和缺陷的关系就越小。
% 如果两个概率分布相差很大,说明这个谓词对应的抽象状态很有可能就有缺陷状态。
% 引入这个缺陷状态的语句很可能就是出错的语句。

除了预定义谓词以外,研究人员还提出各种从程序中获取谓词的方法。
Le等人提出 Savant\parencite{Le2016A} ,使用程序中的不变式的变化来划分状态。
程序中的不变式使用 Daikon\parencite{Ernst2007The} 挖掘。
Savant使用Learning-to-rank方法,通过分析经典的怀疑度分数和在通过的测试用例和失败的测试用例上观察到的不变式,来定位错误的方法。
Savant基于三个出发点。一,在失败的测试用例和通过的测试用例中表现出不同的不变式的程序元素,被怀疑是有错误的。
二,如果这些程序元素拥有很高的经典的怀疑度分数,那么它们更有可能是错误的。
三,有一些不变式比其他不变式更加可疑,比如\mycode{ x == null}。
% 而 Savant 的工作并没有引用 Liblit \parencite{Liblit2005Scalable}和 Liu \parencite{Liu2006Statistical},
% 很可能是在不知道统计性调试的情况下完成的。

\subsection{基于变异的缺陷定位}

变异是对程序的任意随机修改,由变异算子得到。
变异分析是测试领域的一个概念,被用于衡量一个测试集的好坏。
变异分析在程序中插入变异,得到很多变异体,然后使用一组测试去执行变异体。
如果一个测试集中任意测试在一个变异体上得到不同的结果,那么这个变异体被这个测试杀死。
能杀死越多变异体的测试集越好。

变异被引入缺陷定位,用于定位缺陷。
Papadakis等人提出Metallaxis\parencite{Papadakis2015Metallaxis},一个基于变异的缺陷定位。
Metallaxis基于两个假设:
\begin{itemize}
\item 当变异和错误在一个程序的同一条语句上时,失败的测试用例输出发生变化的概率大于通过的测试用例输出发生变化的概率。
\item 当变异和错误不在同一条语句上时,通过测试用例输出发生变化的概率大于失败的测试用例输出发生变化的概率。
\end{itemize}
基于表\ref{mutant_symbol},Metallaxis的怀疑度计算公式为
$$
\mathrm{Metallaxis}(m) = \frac{m_f}{\sqrt{F \times (m_f + m_p)}}
$$
与Ochiai类似。

Moon等人提出另一个基于变异的缺陷定位技术MUSE\parencite{Moon2014Ask}。
MUSE利用变异分析去捕捉单个语句和观察到的缺陷之间的关系。
MUSE基于的两个假设是:
\begin{itemize}
\item 一个失败的测试用例,比起在变异了正确语句的变异体上,在变异了错误语句的变异体上更容易变成通过的。
\item 一个通过的测试用例,比起在变异了失败语句的变异体上,在变异了正确语句的变异体上更容易变成失败的。
\end{itemize}
基于表\ref{mutant_symbol},MUSE的怀疑度计算公式为
$$
\mathrm{MUSE}(m) = m_{f2p} - m_{p2f} \times \frac{\sum_{m}^{}{m_{f2p}}}{\sum_{m}^{}{m_{p2f}}}
$$

\begin{table}
\centering
\caption{基于变异的缺陷定位的数学符号及其意义}
\begin{tabular}{|c|c|}
\hline
$m$ & 变异体 \\
\hline
$m_f$ & 变异m导致输出发生变化的失败的测试用例个数 \\
\hline
$m_p$ & 变异m导致输出发生变化的通过的测试用例个数 \\
\hline
$m_{f2p}$ & 变异m导致失败的测试用例变成通过的测试用例的个数 \\
\hline
$m_{p2f}$ & 变异m导致通过的测试用例变成失败的测试用例的个数 \\
\hline
$F$ & 失败的测试用例的个数 \\
\hline
\end{tabular}
\label{mutant_symbol}
\end{table}

\subsection{基于构造正确执行状态的缺陷定位}

MUSE通过变异体,可以把失败的测试用例变成通过,通过的测试用例变成失败的。
假如有一个变异体,它可以把失败的测试用例变成通过的,且不会影响通过的测试用例,那么这个变异体很可能就是缺陷的补丁。
但是直接分析出这样的变异体是很困难的。

Zhang提出的谓词翻转\parencite{Zhang2006Locating}巧妙地避免了直接分析出正确的补丁,而是使用改变程序状态来达到相同的目的。
假如出错的是一个布尔表达式,改变程序中一个布尔表达式的取值(把真变成假,或者把假变成真),强制改变执行的分支。
假如谓词翻转后,失败的测试用例变成通过的,那么对应的布尔表达式很可能有错误。

谓词翻转是局限在布尔表达式,天使调试\parencite{Chandra2011Angelic}则试图解决任意表达式的错误。
天使调试要求同时具有天使性和灵活性。
天使性是指,存在常量c(天使值)把表达式的求值结果替换成c,失败的测试变得通过。
灵活性是指,对于一个修复的候选$e$,和一个通过测试用例输入$I_p$,如果把$e$
的值替换成一个不同的值(不同于$I_p$下$e$的值),这个测试仍然通过。
利用符号执行约束求解计算得到天使值。
也由于符号执行的开销,天使调试无法应用到大型程序上。

\subsection{基于算法式调试的缺陷定位}

Shapiro提出的算法式调试\parencite{Shapiro1982Algorithmic},通过对子问题询问“是”或“否”来定位缺陷。
算法式调试把复杂的计算步骤拆为小的子问题。
算法式调试的一个问题是,子问题的正确结果可能是不知道的。
如果是让人进行交互式地判断,那么人需要花费时间计算判断子问题的结果。

\subsection{基于差异化调试的缺陷定位}

差异化调试由Zeller等人提出\parencite{Zeller2002Isolating,Zeller2002Simplifying}。
不同于以往的使用动态分析或静态分析的方法去关注源代码,
差异化调试关注程序状态,特别地,差异化调试关注当程序没有出错时的程序状态和程序出错时的程序状态。
差异化调试尝试找到一个最小的修改集合,当把这个集合应用到没有出错时的程序状态后,程序出错了。

\section{机器学习相关工作}

近年来,机器学习相关的工作在高速地发展中。
机器学习也越来越多地被应用到软件工程的领域,并且发挥着重要的作用。
神经网络是一种常用的机器学习模型。
神经网络的定义多种多样,采用 Kohonen 1998年在期刊《Neural Networks》上的定义,为
“神经网络是由具有适应性的简单单元组成的广泛并行互连的网络,它的组织能够模拟生物神经系统对真是世界物体所作出的交互反应”。
本文使用的误差后向传播网络能够近似复杂的非线性函数\parencite{Hecht1992Theory}。
Neumann等人提出一种结合了主成分分析和误差后向传播网络的软件风险分析\parencite{Neumann2002An}。
Tadayon使用神经网络做软件代价评估。
在缺陷定位方面,也有很多神经网络的应用。
比如之前提到的Wong的两篇工作\parencite{W2009BP,Wong2012Effective}。

\section{缺陷定位的数据集}

要研究缺陷定位,需要一个包含缺陷的数据集。
这个数据集一般来说需要有多个缺陷。
对每一个缺陷,会有对应的测试用例,和对应的一个正确的版本。
这些测试用例中既有通过的,也必定有失败的。
失败的这个测试用例就是由缺陷导致。

Siemens数据集\parencite{Hutchins1994Experiments}是一个很早的数据集,用于测试充分性的实验。
它由七个C程序组成,大小在141行到512行之间。
这七个C程序衍生出132个有缺陷的C程序。
每一个错误版本会恰好有一个缺陷。
这个缺陷可能涉及多行甚至多个文件。
但是这些程序的缺陷是由作者手动插入的,根据作者的描述其实和一个简单的变异操作非常相似。
Siemens数据集的输入被构造用于实现完全的代码覆盖。
尽管它一开始并不是被用于缺陷定位,但是很多缺陷定位技术都使用它来验证效果,比如最早的基于频谱的缺陷定位方法\parencite{Renieres2003Fault},Ochiai\parencite{Abreu2006An,Abreu2007On},
SOBER\parencite{Liu2006Statistical},BPNN\parencite{W2009BP}等等。

Defects4j数据集\parencite{Just2014Defects4J}是一个真实、独立、可重现缺陷的数据集。
它的v1.0版本由五个Java开源项目的357个缺陷组成(该数据集仍在更新当中)。
每一个错误版本会恰好有一个缺陷。
这个缺陷可能涉及多行甚至多个文件。
与Siemens数据集相比,~Defects4j~数据集的缺陷和测试用例都更接近实际开发情况。
Savant\parencite{Le2016A}就是在Defects4j数据集上验证的。

	\chapter{研究背景}

\section{缺陷定位例子}

\section{使用基于频谱的缺陷定位}

\section{使用基于状态覆盖的缺陷定位}

比如,对于如下代码C代码,
\lstset{language=C}
\begin{lstlisting}
a = abs(a);
if (update_b) {
    b = sqrt(a);
}
\end{lstlisting}
当\mycode{a}和\mycode{b}的类型都为\mycode{int}时,如果\mycode{a}的值为最小的\mycode{int}时(\mycode{a = -2147483648}),
则代码会在第3行出错(\mycode{b}的值为\mycode{NAN})。
这是因为当\mycode{a = -2147483648}时,第1行的\mycode{a}会被赋值为一个负数,于是在第3行进行\mycode{sqrt}操作的时候,就被出错。
在第1行的时候,考虑两个抽象的状态\mycode{a $\ge$ 0}和\mycode{a $<$ 0}。
发现通过的测试只有\mycode{a $\ge$ 0}这个状态,而失败的测试只有\mycode{a $<$ 0}这个状态。
所以可以认为\mycode{a $<$ 0}是缺陷状态,引入这个状态的第1行的语句很可能就是缺陷语句。
	\chapter{设计}

\section{结合基于频谱的缺陷定位和基于状态覆盖的缺陷定位}
\label{sec:approach_comb}

既然基于频谱的缺陷定位和基于状态覆盖的缺陷定位各有优劣,
那么是否可以结合这两种缺陷定位的方法呢?
事实上已经有研究\parencite{Le2016A,Xuan2014Learning},结合了多种缺陷定位方法,并且获得了比较好的结果。
\todo{展开}。
但是这些研究的结合方式都是在比较高的层次,
比如使用机器学习方法对不同缺陷定位得到的结果进行组合。
这样的结合方式会有两个缺点。
一是他们难以解释为什么他们的方法会起作用。
二是他们没有深入理解缺陷定位方法起作用的原因,仅仅是把各个方法的结果合在一起。

所以,本文试图提出一个能够结合多种缺陷定位(比如基于频谱的缺陷定位和基于状态覆盖的缺陷定位)的方法去改进缺陷定位技术,
同时本文试图解释这个结合为什么起作用的原因。

虽然在直觉上我们认为基于频谱的缺陷定位和基于状态覆盖的缺陷定位是完全不一样的。
因为基于频谱的缺陷定位依靠的是程序元素的覆盖情况,
而基于状态覆盖的缺陷定位依靠的是用谓词来划分状态。
但是事实上这两种缺陷定位技术有相似的地方。
基于频谱的缺陷定位的频谱信息,其实相当于是对每一个语句都关联了一个\mycode{true}这样的谓词。
这样看来基于频谱的缺陷定位相当于基于状态覆盖的缺陷定位的一个特殊情况。
而基于状态覆盖的缺陷定位收集的谓词的覆盖信息也可以看做是程序频谱信息的一种,
所以基于状态覆盖的缺陷定位也可以看做基于频谱的缺陷定位的一个特殊情况。

考虑\ref{sec:state_based}章中基于状态覆盖的缺陷定位的例子。
统计性调试和SOBER都无法给出很好的定位结果。
但是当观察统计性调试的覆盖情况\ref{math_2_return},
我们却可以“猜测”出当前语句很可能是错误语句。
这是因为我们带入了基于频谱的缺陷定位的假设:被失败的测试用例执行的语句,
更有可能有错误。而被通过的测试用例执行的语句,更有可能是正确的。
根据这个假设,表\ref{math_2_return}中的谓词3、4、6都不太可能是能够划分出缺陷状态的谓词,因为它们都没有被失败的测试用例覆盖过。
谓词1最有可能是能够划分出缺陷状态的谓词,其次是谓词2,最后是谓词5。
这是因为谓词1、2、5都被一个失败的测试用例覆盖过,而谓词1没有被通过的测试用例覆盖过。
这种情况下被越少的通过的测试用例覆盖,越有可能就是能够划分出缺陷状态的谓词。
怎样去具体地表示这个怀疑度呢?
这其实是基于频谱的缺陷定位解决的问题了,那就是使用怀疑度公式。
使用Ochiai怀疑度公式去计算表\ref{math_2_return}中谓词的怀疑度,得到表\ref{math_2_ochiai}。
可见谓词1以1.0000的分数远远高于其他谓词,成为怀疑度很大的谓词。
对于每个语句的多个谓词,采用怀疑度最高的谓词的分数作为这个语句的怀疑度。
使用Ochiai怀疑度公式,计算Math的第二个缺陷的各个谓词怀疑度,
错误语句排名第3(第1到4名并列),相比于基于频谱的状态覆盖第11位、统计性调试全部为0和SOBER第10的结果,有显著提升。

不仅如此,这样得到的结合了基于频谱的缺陷定位和基于状态覆盖的缺陷定位的结果,
还可以和原始的基于频谱的缺陷定位的结果相结合。
比如Math的第二个缺陷的错误语句,其基于频谱的缺陷定位的怀疑度为0.3780,
加上结合后的谓词怀疑度1.0000,总怀疑度为1.3780,在怀疑度列表中排名第二。
效果好于原始的定位结果和结合后的结果。

所以,本文提出两种结合基于频谱的缺陷定位和基于状态覆盖的缺陷定位的方式:
\begin{itemize}
\item \textbf{PREDSBFL}:
使用基于频谱的缺陷定位公式去计算谓词的怀疑度,把谓词怀疑度映射到语句,然后根据语句怀疑度定位缺陷。
\item \textbf{COMBSD}:
使用基于频谱的缺陷定位公式去计算语句的怀疑度(原始的基于频谱的缺陷定位方法)得到$s_0$,
然后和 PREDSBFL 计算得到的同一条语句的怀疑度$s_1$结合得到新的怀疑度
$s = (1 - \alpha) \times s_{0} + \alpha \times s_{1}$。
\end{itemize}

\begin{table}
\centering
\begin{tabular}{|c|l|c|}
\hline
 & 谓词 & Ochiai分数\\
\hline
1 & \mycode{retValue < 0} &  1.0000 \\
\hline
2 & \mycode{retValue <= 0} &  0.7071 \\
\hline
3 & \mycode{retValue > 0} & 0.0000 \\
\hline
4 & \mycode{retValue >= 0} & 0.0000 \\
\hline
5 & \mycode{retValue != 0} & 0.4082 \\
\hline
6 & \mycode{retValue == 0} & 0.0000 \\
\hline
\end{tabular}
\caption{使用Ochiai计算谓词怀疑度,其中 \\ \mycode{retValue = (double) (getSampleSize() * getNumberOfSuccesses()) / (double) getPopulationSize()}}
\label{math_2_ochiai}
\end{table}

\section{预定义谓词和预测谓词}

在统计性调试和SOBER中,都使用的是预定义的谓词。
一个谓词的好坏决定了能否划分出缺陷状态。

考虑Defects4j中Math的第四个缺陷,其代码如下:
\lstset{language=Java}
\begin{lstlisting}
public Vector3D intersection(final SubLine subLine, final boolean includeEndPoints) {
    // compute the intersection on infinite line 
    Vector3D v1D = line.intersection(subLine.line);
+   if (v1D == null) {
+       return null;
+   } 

    // check location of point with respect to first sub-line
    Location loc1 = remainingRegion.checkPoint(line.toSubSpace(v1D));
    ... 
}
\end{lstlisting}

第4,5,6行是修复缺陷的代码。这个的缺陷是缺少了对变量\mycode{v1D}是否是空的判断。
考虑谓词\mycode{v1D == null},会发现通过的测试用例都不会覆盖这个谓词,只有失败的测试用例覆盖这个谓词。
因为一旦这个谓词为真,那么后续某些使用这个\mycode{v1D}变量的操作就会造成空指针的错误。

然而这个谓词并不能由预定义的谓词得到。
但是事实上代码中其他的地方存在\mycode{var == null}这样的判断。
于是本文提出了一种基于机器学习的预测谓词的方法,来更准确地找出划分缺陷状态的谓词。

或许我们不使用机器学习方法,而是把\mycode{var == null}这样经典的判断加入预定义谓词呢?
但问题是这样的谓词永远是加不完的。
而且对于每个程序,它们可能还拥有跟自己上下文有关的独特的谓词。
所以从它们自己的代码中来学习出谓词是更有效的方法。

\section{缺陷定位框架}
\label{sec:fl_frame}

本文提出了结合了基于频谱的缺陷定位和基于状态覆盖的缺陷定位的缺陷定位方法PREDSBFL和COMBSD,
同时还提出了一种机器学习方法去预测谓词。

本文的缺陷定位框架主要包括以下几步
\begin{itemize}
\item \textbf{收集特征} 
当使用预定义谓词时不需要这一步。
假如缺陷代码是版本$v$,则从版本$v$的源代码中提取特征。
特征分为两种,一种是变量预测模型的特征,一种是谓词预测模型的特征。
\item \textbf{训练模型}
当使用预定义谓词时不需要这一步。
把得到的特征放入机器学习模型中训练。
两种特征分别单独训练,得到一个预测变量出现在谓词中概率的模型(简称VAR模型),和一个根据一个变量预测可能出现哪些谓词的模型(简称EXPR模型)。
\item \textbf{收集失败测试用例覆盖的语句}
只有被失败的测试用例覆盖的语句才有可能是错误的语句。
因为只需要对失败测试用例覆盖的语句收集其谓词的覆盖情况就可以了。
\item \textbf{获取谓词}
如果使用预定义谓词,则根据预定义谓词的规则分析代码,得到谓词。
如果使用预测谓词模型,则提取需要插入谓词的相关变量和语句的特征,
放入已经训练好的模型中,预测出当前位置最有可能出现的$N$个谓词。
\item \textbf{收集谓词覆盖情况}
把谓词插入代码中,并执行测试用例,收集谓词的覆盖情况。
不同怀疑度公式收集的覆盖情况可能会有不同。
覆盖情况包括:谓词的真(或假)分支被多少个失败(或通过)的测试用例覆盖,
谓词(无论是真分支还是假分支)被多少个失败(或通过)的测试用例覆盖,
谓词的真(或假)分支在一个失败(或通过)的测试用例中被覆盖的次数等等。
\item \textbf{计算怀疑度}
根据上一步收集的谓词覆盖情况,带入基于频谱的缺陷定位和基于状态覆盖的缺陷定位的怀疑度公式计算怀疑度。
收集语句的覆盖情况,带入怀疑度公式计算怀疑度,并且与谓词怀疑度结合。
\end{itemize}

\section{基于机器学习的预测谓词模型}

对谓词的预测分为三步:
\begin{itemize}
\item 第一步,使用VAR模型预测语句中的某个变量出现在谓词中的概率$P_{var}$。
\item 第二步,使用EXPR模型预测语句中的某个变量会出现在谓词 $pred_i$ 中的概率 $P_{pred_i}$ 。
\item 第三步,将谓词按照$P_{var} \times P_{pred_i}$排序。
\end{itemize}

第一步中的变量是赋值表达式的左值。也就是说我们只会对失败的测试用例覆盖的赋值语句进行表达式的预测。

\subsection{机器学习特征}

对于一个有缺陷的代码,我们从当前这个版本的所有源代码中提取特征。

使用的变量特征和谓词特征如表\ref{var_feature}。
不同的是VAR模型的标签是这个变量是否出现在条件中,
EXPR模型的标签是这个变量相关的谓词是什么。
也就是说VAR模型使用的样本来自于所有变量,而EXPR模型使用的样本仅来自于出现在谓词中的变量。

这些特征有的是认为这些特征相似度更高的变量更可能会有相似的属性。
相似的文件名,函数名可能是实现着相似的功能,比如可能是继承了同一个父类的子类。
变量名相似也可能是有相似的功能,比如\mycode{len}和\mycode{length}。
类型相同同理,比如对一个\mycode{Object}类型的变量进行是否为空指针的判断。
相似的功能可能就会使用相似的谓词。
还有特征是描述这个变量的上下文信息,这些特征也会影响谓词。
比如 LastAssign 说明了这个变量值是如何得到的,比如从\mycode{getMin()}函数得到的值\mycode{v}容易出现在谓词中,并且谓词
很可能是\mycode{v < 0}这样的比较。
有的特征直接描述这个变量在条件表达式中的情况,比如 InFor, InCondNum等。


\begin{table}
\centering
\begin{tabular}{|l|l|}
\hline
特征名称 & 特征说明 \\
\hline
FileName & 文件名 \\
\hline
MethodName & 函数名 \\
\hline
VarName & 变量名 \\
\hline
VarType & 变量类型 \\
\hline
LastAssign & 变量最后一次赋值的操作 \\
\hline
Dis0 & 变量声明和使用之间的距离 \\
\hline
PreAssNum & 变量此前赋值的次数 \\
\hline
IsParam & 变量是否是方法的参数 \\
\hline
InFor & 变量是否出现在循环中 \\
\hline
InCondNum & 变量在条件中出现的次数 \\
\hline
BodyUse & 变量在条件语句的语句体里的使用情况 \\
\hline
OutUse & 变量在条件外的使用方式 \\
\hline
\end{tabular}
\caption{VAR模型和EXPR模型特征}
\label{var_feature}
\end{table}

\subsection{机器学习特征编码}

\subsubsection{编码}

对于数值型的变量,直接使用其值作为特征。
而对于分类变量(categorical variable),虽然其值可能表现为0、1、2……
这样的数字,但是其实并不存在$0 < 1 < 2$这样的大小关系。
直接使用其值会让模型以为这个变量存在大小关系。
所以对于分类变量,本文采取独热(one-hot)编码。
比如对于一个值为0、1、2的分类变量$v$,使用新的变量$v_0$、$v_1$、$v_2$代替$v$。
其中$v_i = 1$表示$v = i$,而$v_i = 0$表示$v \ne i$。

\subsubsection{字符串特征编码}

对于字符串类型的特征,比如文件名、函数名和变量名,本文也会使用独热编码。
但是直接使用独热编码会有两个问题:
\begin{enumerate}
\item 特征维度过大。比如对Defects4j中Math项目的第一个缺陷来说,仅不同的方法就有795个。
这意味着如果把方法改成独热编码,将会把一个1维的特征变成795维的特征。
这样可能造成维度灾难\parencite{Richard1957Dynamic}。
虽然一开始随着特征数的上升,机器学习模型的预测效果也会上升,但是当维度过高的时候,实际性能是下降的。
但这还不是最关键的因素。
\item 丢失了字符串内部的特征。
字符串的变量和其它的变量不同,它们之间还有内部的特征。
比如变量名len和length之间的相似度比len和domain之间的相似度要高,
文件名SubLine.java和Line.java之间的相似度比SubLine.java和BracketFinder.java之间的相似度高。
简单地把不同字符串看成完全独立地不同特征会丢失很多有用的信息。
\end{enumerate}

对字符串的编码采取三步。

首先是将字符串转换为向量。对变量名,采取一种类似2-gram的方法。
每一个变量名都会被转成一个长度为$729 (27 \times 27)$的一维向量。
变量名中的大写字母会先被转为小写字符。
对转换后的变量名$s_1s_2...s_n$,考虑每两个相邻字符的字符串$s_1s_2$,$s_2s_3$……$s_{n - 1}s_n$。
这个向量的第i位为1表示变量名中存在两个相邻字符$s_js_{j+1}$,满足$f(s_j) \times 27 + f(s_{j + 1}) = i$。
$f$是一个映射函数,将一个字符转换成一个0到26之间的数字。
对于$f$有$f('a') = 0, f('b') = 1, ..., f('z') = 25$,然后a到z以外的字符都被转为26。
这样的话len和length的特征向量之间就会有两位相同,而len和domain之间则完全没有相同的。
对函数名和文件名,去掉后缀,然后利用Java中使用的驼峰命名法将它们按照驼峰分割开来。
比如SubLine.java会被拆为Sub和Line,而Line.java得到Line,BracketFinder.java得到Bracket和Finder。
收集所有分割后的词(函数名和文件名分开收集),假设有$N_{func}$和$N_{file}$个,
则每个函数名(或文件名)就转成一个长度为$N_{func}$(或$N_{file}$)的向量。
这个向量的第i位为1表示函数名(或文件名)中含有字符串$g_{func}(i)$($g_{file}(i)$)。
$g_{func}$和$g_{file}$是向量下标和字符串的映射。
比如$g_{file}(57) = "Line"$的话,Line.java的向量的第57位为1,其他位为0。

然后是对字符串转换后的向量进行聚类。
聚类使用的是K-means聚类算法。
因为不同的项目所涉及的变量名、函数名、文件名数量差别较大,不适合使用单一的某个常数作为聚类的类别数。
所以聚类的类别数为$Unique(names) / 20$。
其中$names$表示所有变量名或函数名或文件名,$Unique(x)$表示$x$中不重复的值的数量。

最后是根据聚类给字符串编码。
加入聚类结果有$N$个类,那么就把这$N$个类编号为$1$到$N$。
每个类中的每个字符串的编号等于它所处的类的编号。
每个字符创的编号就是它的特征值,对这个特征值使用独热编码就得到最终的特征。
于是通过重新编码和聚类,字符串特征的内部特征被抽取出来,并且字符串特征的维度也被大幅减小。

\subsubsection{预测时的特征新值}

在预测的时候,特征里面可能会有在训练的特征里面没有出现过的值。
如果这个值是数值型的值,那么直接使用这个值就可以。
但是如果是经过编码之后的值,就需要给这个新值一个编码。

为了给出一个和训练时编码一致的编码,需要保存训练时的部分数据。
包括变量名、文件名、函数名的聚类模型,文件名、函数名中向量下标和字符串的对应函数$g_{func}$、$g_{file}$,
训练时使用的独热编码。

对于新的变量名,按照训练时的处理方法将其转为$729 (27 \times 27)$的一维向量。
然后计算出这个向量与训练时变量名聚类模型中的哪个中心点距离更短,把这个变量名划入这个中心点的分类。
最后使用训练时变量名的独热编码给这个分类编码即可。

对于新的文件名或函数名,有两种情况。
第一种情况是,这个文件名虽然没有出现过,但是它按照驼峰拆分之后的字符串都出现过。
这种情况下使用$g_{func}$和$g_{file}$直接构造特征向量。
第二种情况是,这个文件名按照驼峰拆分之后的字符串有没有出现过的。
没有出现过的字符串则会被忽略。
构造出的特征向量使用训练时的聚类模型划分分类,最后使用训练时的独热编码给改分类编码。

对于除变量名、文件名、函数名以外的分类变量,
如果出现了新的值,假设这个分类变量的类别有$C$个,
那么新值转成独热编码后其特征向量为$C$个0。

\subsection{机器学习模型}

本文在小量数据上尝试了多种机器学习模型,最后选定了两种机器学习模型:全连接神经网络和决策树模型。

\subsubsection{神经网络模型}

VAR模型和EXPR模型使用同样的全连接神经网络。
这个神经网络由一个输入层,六个隐藏层,一个输出层,一个softmax层构成。
输入层的神经元数量和特征数量一致,输出层的神经元数量和分类数量一致。
对于VAR模型来说输出层的神经元有两个,对于EXPR模型来说输出层的神经元数量和可能的谓词数量一样。
六个隐藏层每层都是64个节点。这个是小部分实验之后得出的比较适合的值。
采用的激励函数是广泛使用的线性整流函数(ReLU)。
$$
\mathrm{ReLU}(features) = max(features, 0)
$$
神经网络使用误差逆传播算法进行训练,优化算法使用初始学习率为0.05的 Adagrad 算法。
Adagrad 算法能够在训练中自动对学习率进行调整。
损失计算采用的softmax cross entropy。

假如分类数是$c$,softmax层的输入是一个$c \times 1$的向量,输出也是一个$c \times 1$的向量,
表示当前输入的标签是各个类别的概率。
记原始的输入向量是$(a_1, a_2, ..., a_c)$,softmax层的输出是
$(S_1, S_2, ..., S_c)$,则有
$$
S_j = \frac{e^{a_j}}{\sum_{k=1}^c{e^{a_k}}}
$$
以softmax层的输出计算softmax cross enptropy:
$$
E = -\sum_{j = 1}^{c}{y_jlog(S_j)}
$$
其中$y_j$是输入对应的真实的标签。

\subsubsection{决策树模型}

决策树是一种常用的机器学习方法。顾名思义,决策树是一棵树,包含一个根节点、若干个内部节点和叶节点。
每个叶节点对应一个决策结果,内部节点和根节点则对应于一个属性测试。
根据节点的属性测试,样本被划分到对应的子节点中。
其学习流程的基本思想是分治法。

决策树中需要依靠节点的纯度来选择最优划分属性。
本文使用的基尼指数来选择划分属性。
假设当前样本集合为$D$,其中第$k$类样本所占比例为$p_k,(k=1,2,...,|\gamma|)$,
则基尼指数为:
$$
\mathrm{Gini}(D) = \sum_{k=1}^{|\gamma|}{\sum_{k^\prime \ne k}{p_kp_{k^\prime}}} = 1 - \sum_{k = 1}^{|\gamma|}p_k^2
$$
基尼指数反映了从数据集中随机选取两个样本,其类别标记不一致的概率。
所以基尼值越小,纯度越高。
所以选择使得划分后基尼指数最小的属性作为最优划分属性。

树的深度没有限制,也就是说节点会被一直展开直到所有叶节点都是纯净的节点(即所包含的样本都只属于同一个分类)或叶节点的样本数量小于2。

\section{收集频谱信息}

代码插装是一种常用的记录程序执行情况、修改程序的方法。
采用的工具是eclipse提供的Java Development Tool\footnote{\url{http://www.eclipse.org/jdt/}},简称JDT。
JDT不仅可以构造给定Java代码的抽象语法树,还可以新建、修改、插入、删除抽象语法树从而新建、修改、插入、删除Java代码。
本章中很多实现都依赖于JDT。

本文使用代码插装技术主要目的是收集频谱信息。
比如为了收集失败测试用例覆盖的语句,可以在每个语句后面加上一个打印语句。
这个打印语句会打印出当前的文件名和对应语句的行数。
用失败的测试用例执行这个插装后的程序,就可以得到失败测试用例覆盖的语句。

对于谓词$P$和一个测试用例$T_i$,可能需要收集种信息:
\begin{enumerate}
\item $P$是否被$T$观察过(无论是真还是假)。
\item $T_i$中$P$是否至少一次为真。
\item $T_i$中$P$为真的次数。
\item $T_i$中$P$为假的次数。
\end{enumerate}

对于基于频谱的缺陷定位的公式来说,需要第2个信息。
对于统计性调试的公式来说,需要第1、2个信息。
对于SOBER的公式来说,需要第3、4个信息。
所以整体来说谓词的插装有两种形式。
第一种用于基于频谱的缺陷定位公式和统计性调试公式:
\lstset{language=Java}
\begin{lstlisting}
// predicateSignature is a string that uniquely represents the predicate.
SpecLogger.observe(predicateSignature);
if (predicate) {
    SpecLogger.cover(predicateSignature);
}
\end{lstlisting}
\mycode{SpecLogger.observe}记录这个谓词为观察过(信息1),
\mycode{SpecLogger.cover}记录这个谓词为真过(信息2)。
第二种用于SOBER公式:
\lstset{language=Java}
\begin{lstlisting}
// predicateSignature is a string that uniquely represents the predicate.
if (predicate) {
    SpecLogger.coverTrueBranch(predicateSignature);
} else {
    SpecLogger.coverFalseBranch(predicateSignature);
}
\end{lstlisting}
\mycode{SpecLogger.coverTrueBranch}记录这个谓词为真的次数(信息3),
\mycode{SpecLogger.coverFalseBranch}记录这个谓词为假的次数(信息4)。

这几个$SpecLogger$的函数可以是把自己内部的某个布尔型的成员变量置为真或假,
也可以是把自己内部的某个整型的成员变量加一。
这个$SpecLogger$对象在每个测试用例开始的时候重置(使用静态方法变量),并在测试用例结束的时候以某种格式将自己记录的信息输出到文件。
\lstset{language=Java}
\begin{lstlisting}
private void testSomething() {
	SpecLogger.reset();
    SpecLogger.testStatus = true;

    // test ...
    ...

    SpecLogger.dump();
}
\end{lstlisting}
第三行是根据当前测试用例\mycode{testSomething}是通过的测试用例还是失败的测试用例而插装的。
最后\mycode{SpecLogger.dump}把记录的信息输出到文件。
其他程序从这个输出文件中可以构造出频谱信息。

\section{不改变程序状态地插入谓词}

统计性调试中自定义的谓词和预测出来的谓词可能有副作用。
使用上一章中的插装方法会改变程序的执行状态,所以需要不改变程序状态地插入谓词。

统计性调试中的谓词要么是两个变量构成的二元表达式,要么就是本来就会在程序中执行的条件表达式。
前者不会有副作用,而后者即使有副作用,由于其本来就要在原程序中执行一次,因此只要利用执行的这一次的结果就可以。

预测出来的谓词可能含有有副作用的函数,或者含有赋值语句。
于是对预测出谓词的静态分析,只留下一定无副作用的谓词。
比如除了部分指定函数(如size),含有其他函数的谓词都会被过滤掉。

插入谓词仍然使用JDT。

\subsection{插入预测的谓词}

通过机器学习模型,每个语句可能会关联一组谓词。
这些谓词会被机器学习模型赋予出现的概率。
最终每个语句我们选择概率最高的5个谓词作为需要插入的谓词。

在插入预测的谓词之前,要先对谓词进行过滤。
过滤包括两步:
\begin{enumerate}
\item 静态分析过滤掉可能不合法的谓词(比如对一个\mycode{int}类型的变量进行下标访问)。
\item 静态分析过滤掉可能有副作用的谓词。
\item 过滤掉不能编译的谓词。
\end{enumerate}

一个预测出来的谓词分为两部分,一个是谓词$P(x)$,一个是变量$v$,最终构成谓词$P(v)$。
一个谓词会被判定为不合法如果它满足以下至少一点:
\begin{itemize}
\item 含有数组访问\mycode{v[i]}且\mycode{v}并不是数组类型。
\item 含有变量访问\mycode{v.a}且\mycode{v}是一个基本数据类型(比如\mycode{int})的变量。
\todo{bug fix}
\item 含有变量访问\mycode{a.v}且\mycode{v}不是\mycode{a}的一个域。
\item \todo{simple name}
\item 含有函数调用\mycode{v.a()}且\mycode{v}是一个基本数据类型变量。
\item 含有函数调用\mycode{f()}且\mycode{f}不属于预定义的合法函数(如\mycode{size},\mycode{length},\mycode{toString},\mycode{contains},\mycode{containsKey},\mycode{Math.abs},\mycode{Double.isInfinite},\mycode{Double.isNaN})。
\item 含有域访问\mycode{v.a}且\mycode{v}是一个基本数据类型的变量。
\item 含有中缀表达式\mycode{a op b}且\mycode{a,b}的类型和运算符\mycode{op}不匹配(比如对非数字类型进行加法)。
\end{itemize}

过滤有副作用的谓词主要是过滤掉以下几种:
\begin{itemize}
\item 含有前缀表达式,且其中运算符为\mycode{++}或\mycode{--}。
\item 含有后缀表达式。
\item 含有赋值语句。
\end{itemize}

最后单独地插入每一条谓词,过滤掉不能顺利编译的。
在没有判定谓词是否合法的时候,也可以通过编译来排除不合法的谓词。
但是由于谓词数量较多,通过预处理去掉部分肯定不对的谓词可以加速整个流程。

最后对于一个语句$s$,我们可以得到一组合法无副作用的谓词${P_1,P_2...}$。
我们再加入取反的谓词${!P_1, !P_2, ...}$。

然后就是将收集谓词频谱信息的代码插入到语句前或后。
语句前后都插入的有\mycode{WhileStatement,ForStatement,DoStatement,EnhancedForStatement},
插入在语句后面的有\mycode{Assignment,VariableDeclarationStatement, ConstructorInvocation, SuperConstructorInvocation},
插入在语句前的有\mycode{IfStatement,SwitchStatement}等其他所有语句。

\subsection{插入预定义的谓词}

预定义的谓词也可能有副作用,比如赋值语句\mycode{a[b++] = c}的谓词\mycode{a[b++] > d},
如果使用此前的插装方法,则插入\mycode{if (a[b++] > d)}这样的语句会产生副作用。
但是如果使用中间变量,则上述代码可以重写为:
\lstset{language=Java}
\begin{lstlisting}
temp = c;
a[b++] = temp;
if (temp > d) {
    ...
}
\end{lstlisting}
所以预定义谓词的三种情况,分支、返回、数值对,都可以使用中间变量这样的方式来插入谓词,从而避免了副作用的情况。

由于分支对应多种情况,使用中间变量会比较复杂。
分支的谓词是分支中含有的条件表达式。
\mycode{DoStatement}中的条件表达式的值每次循环都会更新,再考虑上\mycode{continue}这样的语句,
会让条件表达式的更新逻辑非常复杂。
为了简单地插入谓词,使用一个函数\mycode{logConditionCoverage}替换条件表达式。
原代码为:
\lstset{language=Java}
\begin{lstlisting}
// example.java:
while(!iter.isEmpty()) {
	...
}
\end{lstlisting}
插装了收集谓词频谱信息的语句后,代码被修改为:
\lstset{language=Java}
\begin{lstlisting}
// example.java
while(SpecLogger.logConditionCoverage(!iter.isEmpty(), "!iter.isEmpty()", "!(!iter.isEmpty())") {
	...
}

// SpecLogger.java
public static boolean logConditionCoverage(boolean condition, String trueLogInfo, String falseLogInfo) {
	observe(trueLogInfo);
	observe(falseLogInfo);
	if (condition) {
		cover(trueLogInfo);
	} else {
		cover(falseLogInfo);
	}
	return condition;
}
\end{lstlisting}
这里同时记录了假分支的覆盖情况。这样做有两个目的:
\begin{itemize}
\item 统计性调试中预定义的分支型谓词有条件表达式取反。
\item 和预测谓词加入了谓词取反的情况保持一致。
\end{itemize}

\section{计算怀疑度}

收集频谱信息后,就可以带入计算怀疑度。
无论是机器学习得到的谓词,还是预定义的谓词,都使用表\ref{susp_formula}中的五种公式,以及统计性调试和SOBER总共七种公式进行计算。
通过谓词计算得到的语句$s_i$的怀疑度记为$s_{i1}$。

除了能够使用谓词计算得到怀疑度以外,利用原始的
基于频谱的缺陷定位方法和基于状态覆盖的缺陷定位方法(SOBER除外),还可以计算得到语句的怀疑度$s_{i0}$。
本文尝试结合这两种怀疑度,使用$s_i = (1 - \alpha) \times s_{i0} + \alpha \times s_{i1}$作为最终怀疑度。


	% 结论。
	% Copyright (c) 2014,2016 Casper Ti. Vector
% Public domain.

\specialchap{结论}
% \pkuthssffaq % 中文测试文字。

% vim:ts=4:sw=4


	% 正文中的附录部分。
	\appendix
	% 排版参考文献列表。bibintoc 选项使“参考文献”出现在目录中;
	% 如果同时要使参考文献列表参与章节编号,可将“bibintoc”改为“bibnumbered”。
	\printbibliography[heading = bibintoc]
	% \bibliographystyle{IEEEtran}%

	% \bibliography{thesis.bib}
	% 各附录。
	% Copyright (c) 2014,2016 Casper Ti. Vector
% Public domain.

\chapter{附件}
% \pkuthssffaq % 中文测试文字。

% vim:ts=4:sw=4


	% 以下为正文之后的部分,默认不进行章节编号。
	\backmatter
	% 致谢。
	\chapter{致谢}

感谢北京大学和北京大学软件工程所,让我能够接触到前沿的科研项目。
浓厚的学术氛围感染了我,敦促我不断学习,打下了科研的坚实基础。

感谢熊英飞研究员,张路教授和郝丹副教授对我的指导。
我从大三开始就在张路老师的小组里学习,近五年的时间里三位老师对我耐心地指导和帮助,让我受益颇多。
从最开始的浮点数计算误差相关的研究,到自动缺陷定位的研究,
我从一个初出茅庐的计算机专业低年级学生,
变成了一个能够完成很多艰难计算机任务的高年级研究生。
老师们的培养让我在计算机基础知识,代码能力,科研能力等方面都有提升。
我也在熊英飞研究和张路教授的指导下发表了两篇CCF-A类的论文,一篇第一作者,一篇第二作者。

感谢姜佳君同学,和我一起讨论、完成这个研究。
他提出了许多宝贵的想法与建议,并且与我一起实现了\textsc{LinSD}。

感谢王博同学和臧琳飞师姐,他们的基于机器学习的缺陷修复给了本文非常多的帮助。
本文使用的从Java代码中提取特征的JDT代码来自于他们的基于机器学习的缺陷修复的代码。

最后,感谢我的父母一直陪伴着我、支持着我。他们一直是我坚强的后盾。
	% 原创性声明和使用授权说明。
	% Copyright (c) 2008-2009 solvethis
% Copyright (c) 2010-2017 Casper Ti. Vector
% All rights reserved.
%
% Redistribution and use in source and binary forms, with or without
% modification, are permitted provided that the following conditions are
% met:
%
% * Redistributions of source code must retain the above copyright notice,
%   this list of conditions and the following disclaimer.
% * Redistributions in binary form must reproduce the above copyright
%   notice, this list of conditions and the following disclaimer in the
%   documentation and/or other materials provided with the distribution.
% * Neither the name of Peking University nor the names of its contributors
%   may be used to endorse or promote products derived from this software
%   without specific prior written permission.
%
% THIS SOFTWARE IS PROVIDED BY THE COPYRIGHT HOLDERS AND CONTRIBUTORS "AS
% IS" AND ANY EXPRESS OR IMPLIED WARRANTIES, INCLUDING, BUT NOT LIMITED TO,
% THE IMPLIED WARRANTIES OF MERCHANTABILITY AND FITNESS FOR A PARTICULAR
% PURPOSE ARE DISCLAIMED. IN NO EVENT SHALL THE COPYRIGHT HOLDER OR
% CONTRIBUTORS BE LIABLE FOR ANY DIRECT, INDIRECT, INCIDENTAL, SPECIAL,
% EXEMPLARY, OR CONSEQUENTIAL DAMAGES (INCLUDING, BUT NOT LIMITED TO,
% PROCUREMENT OF SUBSTITUTE GOODS OR SERVICES; LOSS OF USE, DATA, OR
% PROFITS; OR BUSINESS INTERRUPTION) HOWEVER CAUSED AND ON ANY THEORY OF
% LIABILITY, WHETHER IN CONTRACT, STRICT LIABILITY, OR TORT (INCLUDING
% NEGLIGENCE OR OTHERWISE) ARISING IN ANY WAY OUT OF THE USE OF THIS
% SOFTWARE, EVEN IF ADVISED OF THE POSSIBILITY OF SUCH DAMAGE.

{
	\ctexset{section = {
		format+ = {\centering}, beforeskip = {40bp}, afterskip = {15bp}
	}}

	% 学校书面要求本页面不要页码,但在给出的 Word 模版中又有页码且编入了目录。
	% 此处以 Word 模版为实际标准进行设定。
	\specialchap{北京大学学位论文原创性声明和使用授权说明}
	\mbox{}\vspace*{-3em}
	\section*{原创性声明}

	本人郑重声明:
	所呈交的学位论文,是本人在导师的指导下,独立进行研究工作所取得的成果。
	除文中已经注明引用的内容外,
	本论文不含任何其他个人或集体已经发表或撰写过的作品或成果。
	对本文的研究做出重要贡献的个人和集体,均已在文中以明确方式标明。
	本声明的法律结果由本人承担。
	\vskip 1em
	\rightline{%
		论文作者签名:\hspace{5em}%
		日期:\hspace{2em}年\hspace{2em}月\hspace{2em}日%
	}

	\section*{%
		学位论文使用授权说明\\[-0.33em]
		\textmd{\zihao{5}(必须装订在提交学校图书馆的印刷本)}%
	}

	本人完全了解北京大学关于收集、保存、使用学位论文的规定,即:
	\begin{itemize}
		\item 按照学校要求提交学位论文的印刷本和电子版本;
		\item 学校有权保存学位论文的印刷本和电子版,
			并提供目录检索与阅览服务,在校园网上提供服务;
		\item 学校可以采用影印、缩印、数字化或其它复制手段保存论文;
		\item 因某种特殊原因需要延迟发布学位论文电子版,
			授权学校在 $\Box$\nobreakspace{}一年 /
			$\Box$\nobreakspace{}两年 /
			$\Box$\nobreakspace{}三年以后在校园网上全文发布。
	\end{itemize}
	\centerline{(保密论文在解密后遵守此规定)}
	\vskip 1em
	\rightline{%
		论文作者签名:\hspace{5em}导师签名:\hspace{5em}%
		日期:\hspace{2em}年\hspace{2em}月\hspace{2em}日%
	}

	% 若需排版二维码,请将二维码图片重命名为“barcode”,
	% 转为合适的图片格式,并放在当前目录下,然后去掉下面 2 行的注释。
	%\vfill\noindent
	%\includegraphics[height = 5em]{barcode}
}

% vim:ts=4:sw=4

\end{document}

% vim:ts=4:sw=4
